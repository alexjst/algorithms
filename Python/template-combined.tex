% !TEX program = pdflatex
\documentclass[11pt]{article}
\usepackage[margin=1in]{geometry}
\usepackage[T1]{fontenc}
\usepackage[utf8]{inputenc}
\usepackage{lmodern}
\usepackage{microtype}
\usepackage{parskip}
\usepackage{hyperref}
\usepackage{titlesec}
\usepackage{enumitem}
\usepackage{booktabs}
\usepackage{xcolor}
\usepackage{amsmath,amssymb}
\usepackage{caption}
\usepackage{tcolorbox}

% For beautiful syntax highlighting
\usepackage{minted} % requires pygments + --shell-escape

% ---- Printer-friendly colors (grayscale fallback) ----
\definecolor{codebg}{RGB}{250,250,250}
\definecolor{keyword}{RGB}{0,0,120}
\definecolor{comment}{RGB}{90,110,90}
\definecolor{string}{RGB}{120,20,20}

% ---- Minted settings ----
\setminted[python]{
  fontsize=\small,
  linenos,
  numbersep=6pt,
  frame=lines,
  framesep=2mm,
  baselinestretch=1.1,
  bgcolor=codebg,
  breaklines,
  breakanywhere,
}

% ---- Section formatting ----
\titleformat{\section}{\Large\bfseries}{}{0pt}{}
\titleformat{\subsection}{\large\bfseries}{}{0pt}{}
\titleformat{\subsubsection}{\normalsize\bfseries\itshape}{}{0pt}{}

\hypersetup{
  colorlinks=true,
  linkcolor=black,
  urlcolor=black,
  citecolor=black,
  pdfauthor={Technical Interview Preparation Assistant},
  pdftitle={Coding Interview Patterns & Templates (Python)},
  pdfsubject={Printable reference guide for coding interviews},
}

\begin{document}

\begin{center}
  {\LARGE \textbf{Coding Interview Patterns}}\\[0.4em]
  {\large Python Templates for Muscle Memory}\\[0.4em]
  {\normalsize Version: \today}
\end{center}

\vspace{0.5em}
This reference combines the **printer-friendly focus** of the GPT-5 version, the **beautiful code formatting** of the Gemini version, and the **boxed summaries/tables** of the Claude version. Use the boxed notes for quick recognition, and the templates for daily rewriting drills.

\tableofcontents
\newpage

% ---------------- Example Template ----------------
\section{Core Patterns \,\& \,Templates}

\subsection{Depth-First Search (DFS)}
\begin{tcolorbox}[colback=codebg,colframe=black!50,title=When to Use DFS]
  Use when exploring connectivity, components, cycle detection, or recursion on trees/graphs. Complexity $O(V+E)$. 
\end{tcolorbox}

\subsubsection*{Recursive DFS}
\begin{minted}{python}
from typing import List, Dict, Set

def dfs_recursive(graph: Dict[int, List[int]], start: int) -> List[int]:
    visited: Set[int] = set()
    order: List[int] = []

    def dfs(u: int) -> None:
        if u in visited:
            return
        visited.add(u)
        order.append(u)
        for v in graph.get(u, []):
            dfs(v)

    dfs(start)
    return order
\end{minted}

\subsubsection*{Iterative DFS}
\begin{minted}{python}
from typing import List, Dict

def dfs_iterative(graph: Dict[int, List[int]], start: int) -> List[int]:
    stack: List[int] = [start]
    visited = set()
    order: List[int] = []

    while stack:
        u = stack.pop()
        if u in visited:
            continue
        visited.add(u)
        order.append(u)
        for v in reversed(graph.get(u, [])):
            if v not in visited:
                stack.append(v)
    return order
\end{minted}

% ---------------- Pattern Guide Example ----------------
\section{Pattern Recognition Guide}
\begin{tcolorbox}[colback=codebg,colframe=black!60,title=Trigger Words]
  \begin{itemize}[leftmargin=1.2em]
    \item ``longest/shortest subarray/substring'' $\Rightarrow$ Sliding Window
    \item ``sorted array'' $\Rightarrow$ Two Pointers / Binary Search
    \item ``dynamic connectivity'' $\Rightarrow$ Union-Find
    \item ``prerequisites ordering'' $\Rightarrow$ Topological Sort
    \item ``prefix/suffix queries'' $\Rightarrow$ Prefix Sum / Trie
  \end{itemize}
\end{tcolorbox}

% ---------------- Schedule ----------------
\section{4-Week Study Schedule}
\begin{tcolorbox}[colback=codebg,colframe=black!60,title=Overview]
  Week 1: Fundamentals (Two Pointers, Sliding Window, Binary Search, Prefix Sum, Kadane) \\
  Week 2: Graphs (DFS, BFS, Topological Sort, Union-Find, Heaps) \\
  Week 3: DP \,\& Backtracking \\
  Week 4: Advanced (Trie, Monotonic Stack, Cyclic Sort, Mixed Drills)
\end{tcolorbox}

% ---------------- Daily Routine ----------------
\section{Daily Practice Routine}
\begin{tcolorbox}[colback=codebg,colframe=black!60,title=45--90 Minutes]
  \begin{enumerate}[leftmargin=1.2em]
    \item Warm-up: Rewrite 1--2 templates from memory.
    \item Focused Drill: Solve 2 problems of today’s pattern.
    \item Flash Review: Recap 3 old problems aloud.
    \item Retrospective: Write down mistakes and triggers.
  \end{enumerate}
\end{tcolorbox}

% ---------------- Complexity Notes ----------------
\section{Time \,\& \,Space Complexity}
\begin{tcolorbox}[colback=codebg,colframe=black!60,title=Summary]
  DFS/BFS: $O(V+E)$ time, $O(V)$ space. \\
  Binary Search: $O(\log n)$ time. \\
  Sliding Window: $O(n)$ time, small space. \\
  DP: $O(n)$ or $O(mn)$ depending on states. \\
  Union-Find: $\alpha(n)$ amortized. \\
  Heaps: $O(\log n)$ push/pop. \\
  Trie: $O(L)$ per word. \\
  Monotonic Stack: $O(n)$.
\end{tcolorbox}

% (Other patterns/templates would follow here, all using minted + tcolorbox style)

\end{document}

