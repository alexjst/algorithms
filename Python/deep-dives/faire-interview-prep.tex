% Faire Technical Interview Preparation Guide
% Comprehensive guide for Software Engineer interviews at Faire

\documentclass[11pt,letterpaper]{article}
\usepackage[margin=1in]{geometry}
\usepackage{amsmath}
\usepackage{amssymb}
\usepackage{listings}
\usepackage{xcolor}
\usepackage{hyperref}
\usepackage{enumitem}
\usepackage{titlesec}
\usepackage{fancyhdr}

% Code listing style
\lstset{
    language=Python,
    basicstyle=\ttfamily\small,
    keywordstyle=\color{blue},
    commentstyle=\color{gray},
    stringstyle=\color{red},
    numbers=left,
    numberstyle=\tiny\color{gray},
    stepnumber=1,
    numbersep=5pt,
    backgroundcolor=\color{white},
    showspaces=false,
    showstringspaces=false,
    showtabs=false,
    frame=single,
    tabsize=2,
    captionpos=b,
    breaklines=true,
    breakatwhitespace=false,
    escapeinside={(*@}{@*)},
    morekeywords={self}
}

% Section formatting
\titleformat{\section}{\Large\bfseries}{\thesection}{1em}{}
\titleformat{\subsection}{\large\bfseries}{\thesubsection}{1em}{}

% Header/Footer
\pagestyle{fancy}
\fancyhf{}
\rhead{Faire Interview Prep}
\lhead{\leftmark}
\cfoot{\thepage}

\title{\textbf{Faire Technical Interview Preparation Guide} \\
\large Complete Prep for Software Engineer Roles}
\author{Interview Focus: Algorithms, System Design (Search), ML Basics}
\date{\today}

\begin{document}

\maketitle
\tableofcontents
\newpage

% ============================================================================
\section{Executive Summary: About Faire}
% ============================================================================

\subsection{Company Overview}

\textbf{Faire} is an online wholesale marketplace that connects independent retailers with unique brands. Founded in 2017 and based in San Francisco, Faire uses machine learning to match local retailers with brands and products that uniquely fit their stores.

\textbf{Key Value Propositions:}
\begin{itemize}
    \item \textbf{For Retailers:} Discover unique products, net 60 payment terms, free returns on opening orders
    \item \textbf{For Brands:} Access to 700,000+ retailers globally, data-driven insights, streamlined operations
    \item \textbf{Two-Sided Marketplace:} Balance retailer needs (relevance, conversion, personalization) with brand needs (order volume, exposure)
\end{itemize}

\subsection{Core Technical Challenges}

Based on Faire's product and engineering blog, their main technical challenges include:

\begin{enumerate}
    \item \textbf{Search \& Discovery:} Help retailers find relevant products among millions of SKUs
    \item \textbf{Recommendations:} Inspire retailers in "browsing mode" with personalized suggestions
    \item \textbf{Ranking:} Balance multiple stakeholders (retailers, brands, platform) in ranking algorithms
    \item \textbf{Two-Sided Optimization:} Optimize for both retailer satisfaction and brand success
    \item \textbf{Cold Start:} New retailers and new brands need quality recommendations immediately
    \item \textbf{Real-Time Features:} Low-latency feature serving for ranking models
\end{enumerate}

% ============================================================================
\section{Faire Tech Stack \& Architecture}
% ============================================================================

Understanding Faire's actual tech stack will help you tailor your answers and ask informed questions.

\subsection{Backend \& Languages}

\textbf{Primary Language: Kotlin}
\begin{itemize}
    \item Entire backend monolith written in Kotlin (migrated from Java)
    \item Handles 1,000+ requests per second
    \item Android app also in Kotlin
    \item Allows for concise, type-safe code with null safety
\end{itemize}

\textbf{Why Kotlin?}
\begin{itemize}
    \item Interoperable with existing Java libraries (Hibernate for ORM, Guice for DI)
    \item Modern language features (coroutines for async, data classes)
    \item Strong community and tooling support
\end{itemize}

\subsection{Core Infrastructure}

\textbf{Data Storage:}
\begin{itemize}
    \item \textbf{MySQL:} Primary relational database
    \item \textbf{Redis:} Caching layer and feature store for ML models
    \item \textbf{NoSQL:} For specific use cases
\end{itemize}

\textbf{Search:}
\begin{itemize}
    \item \textbf{ElasticSearch:} Product search with custom scoring functions
    \item \textbf{Lucene:} Underlying search library
\end{itemize}

\textbf{Message Queues \& Streaming:}
\begin{itemize}
    \item \textbf{Amazon SQS:} Asynchronous job processing
    \item \textbf{Kinesis Firehose:} Event streaming to data warehouse
\end{itemize}

\textbf{Compute \& Orchestration:}
\begin{itemize}
    \item \textbf{Kubernetes:} Container orchestration
    \item \textbf{Nginx:} Load balancing and routing
    \item \textbf{AWS ELB:} External load balancing
\end{itemize}

\subsection{Machine Learning Stack}

\textbf{Models:}
\begin{itemize}
    \item \textbf{XGBoost:} Primary model for ranking (search and recommendations)
    \item \textbf{Embedding Models:} Learn representations from user behavior (clicks, add-to-cart, orders)
\end{itemize}

\textbf{Feature Engineering:}
\begin{itemize}
    \item \textbf{Redis Feature Store:} Hundreds of features retrieved in real-time per product
    \item Features include: retailer-product similarity, engagement signals, brand metadata, product attributes
\end{itemize}

\textbf{Data Infrastructure:}
\begin{itemize}
    \item \textbf{DBT:} Data transformation and modeling
    \item \textbf{Tableau:} Business intelligence and analytics
\end{itemize}

\subsection{Two-Stage Ranking Architecture}

Faire uses a classic retrieval + ranking pipeline:

\textbf{Stage 1: Retrieval (Candidate Generation)}
\begin{itemize}
    \item ElasticSearch retrieves top 1,000 candidate products
    \item Uses custom scoring function balancing precision/recall
    \item Fast: optimized for low latency
\end{itemize}

\textbf{Stage 2: Ranking (Re-ranking)}
\begin{itemize}
    \item Retrieve hundreds of features per product from Redis
    \item Score 1,000 candidates with XGBoost model
    \item Rank products by predicted engagement/conversion
\end{itemize}

% ============================================================================
\section{OFFICIAL Interview Structure (From Recruiter)}
% ============================================================================

\subsection{Virtual Onsite - 5 Parts Over 2 Days}

This is the OFFICIAL interview structure provided by Faire's recruiting team for the Back-end Engineer - Search role.

\subsubsection{Day 1: Technical Deep Dive (3 Hours Total)}

\textbf{Part 1: Programming Exercise [1 HOUR]}
\begin{itemize}
    \item \textbf{Format:} Paired programming on CoderPad
    \item \textbf{Goal:} Write a working solution to a coding problem
    \item \textbf{Evaluation Criteria:}
    \begin{enumerate}
        \item \textit{Approach and Planning:} Ask clarifying questions before coding
        \item \textit{Code Structure \& Cleanliness:} Write clean, readable code
        \item \textit{Coachability:} Modify solution based on interviewer feedback
        \item \textit{Debugging:} Account for edge cases and fix bugs
        \item \textit{Testing:} Write test cases with good coverage
    \end{enumerate}
\end{itemize}

\textbf{Part 2: Systems Design - Search [1 HOUR]}
\begin{itemize}
    \item \textbf{Focus:} Major components of an e-commerce search system
    \item \textbf{Topics:}
    \begin{itemize}
        \item How data gets into the system, how it's processed, and how it's queried
        \item Life of a product record from creation to search results
        \item Life of a search query from user input to displayed results
        \item Design tradeoffs and performance optimizations
        \item Scaling considerations
        \item High-level search algorithm concepts
    \end{itemize}
    \item \textbf{Important:} This is \textbf{NOT} about super-detailed ML knowledge
    \item \textbf{Focus:} Architecture, data flows, and system components
\end{itemize}

\textbf{Part 3: Algorithm Interview [1 HOUR]}
\begin{itemize}
    \item \textbf{CRITICAL:} This is \textbf{NOT} a LeetCode-style coding interview!
    \item \textbf{Format:} 1-2 questions with sub-questions
    \item \textbf{NO CODE REQUIRED} (emphasized in official document)
    \item \textbf{Goal:} Evaluate practical design knowledge for ML-powered systems
    \item \textbf{What You'll Do:}
    \begin{itemize}
        \item Identify performance bottlenecks in EXISTING systems
        \item Propose incremental improvements
        \item Suggest architectural changes
        \item Think about ML system improvements (NOT design from scratch)
    \end{itemize}
    \item \textbf{Example:} "Here's our current search ranking system. It has latency issues. What would you investigate and how would you improve it?"
\end{itemize}

\subsubsection{Day 2: Values \& Leadership (1.5 Hours Total)}

\textbf{Part 4: Values Interview [30 MINUTES]}
\begin{itemize}
    \item \textbf{Format:} Behavioral questions
    \item \textbf{Focus:} Alignment with Faire's 5 core values
    \item \textbf{Example Topics:}
    \begin{itemize}
        \item Tell me about a complex problem you solved
        \item Describe a positive/challenging working relationship
        \item When did you think outside the box?
        \item Why Faire? Why our mission?
        \item Examples of impact, influence, ownership
        \item How do you handle feedback and growth?
    \end{itemize}
\end{itemize}

\textbf{Part 5: Levelling Deep Dive [1 HOUR]}
\begin{itemize}
    \item \textbf{Format:} Technical leadership discussion
    \item \textbf{Focus:} Career history, technical depth, seniority calibration
    \item \textbf{Topics:}
    \begin{itemize}
        \item Most complex system you've designed
        \item Technical decisions and tradeoffs
        \item How you've led projects or mentored others
        \item Scaling challenges you've faced
        \item Your technical growth trajectory
    \end{itemize}
\end{itemize}

\subsection{Faire's Core Values (OFFICIAL)}

Understanding these values is CRITICAL for the Values Interview (Part 4).

\subsubsection{1. Be an Owner}
\begin{itemize}
    \item \textbf{Resourceful:} Find creative solutions with available resources
    \item \textbf{Raise the Bar:} Set and maintain high standards
    \item \textbf{Share Our Legos:} Build reusable systems, share knowledge
\end{itemize}

\subsubsection{2. Be Kind}
\begin{itemize}
    \item \textbf{Collaboration:} Work effectively with others
    \item \textbf{Coachability:} Accept feedback and grow
    \item \textbf{Inclusive:} Create welcoming environment for all
\end{itemize}

\subsubsection{3. Embrace the Adventure}
\begin{itemize}
    \item \textbf{Agility:} Adapt quickly to change
    \item \textbf{Bias Towards Action:} Move fast, iterate, learn
\end{itemize}

\subsubsection{4. Seek the Truth}
\begin{itemize}
    \item \textbf{Decision-Making:} Make data-driven decisions
    \item \textbf{Analytical Thinking:} Dig deep to understand root causes
    \item \textbf{Communication:} Communicate clearly and transparently
\end{itemize}

\subsubsection{5. Serve Our Community}
\begin{itemize}
    \item \textbf{Customer Centricity:} Put retailers and brands first
    \item \textbf{Prioritization:} Focus on highest-impact work
    \item \textbf{Motivation:} Inspired by Faire's mission
\end{itemize}

\subsection{Company Mission \& Scale}

\textbf{Mission:} Empower entrepreneurs everywhere to chase their dreams.

\textbf{Scale:}
\begin{itemize}
    \item 350,000+ independent retailers on the platform
    \item Independent retailers collectively larger than Amazon + Walmart combined
    \item 40\% of wholesale buying moved online during COVID
\end{itemize}

\textbf{Growth Flywheel:}
\begin{enumerate}
    \item Brand onboards → Refers other retailers
    \item Retailer onboards → Refers other brands
    \item Network effects create virtuous cycle
\end{enumerate}

\textbf{Data Flywheel:}
\begin{enumerate}
    \item Better recommendations → Higher sell-through for retailers
    \item Higher sell-through → More orders placed
    \item More orders → More data for ML models
    \item Better data → Better recommendations (cycle continues)
\end{enumerate}

% ============================================================================
\section{Programming Exercise Prep (Interview Part 1)}
% ============================================================================

\textbf{Context:} This section covers the Programming Exercise (Part 1 of the interview), which is a 1-hour paired programming session on CoderPad where you write actual code.

\textbf{DO NOT CONFUSE with:} The "Algorithm Interview" (Part 3), which does NOT require code!

\subsection{Expected Difficulty}

\textbf{Format:} Medium LeetCode-level problem, 45-60 minutes

\textbf{Focus Areas:} Arrays, hashmaps, two pointers, trees, graphs, dynamic programming

\textbf{Evaluation:} Approach, code cleanliness, coachability, debugging, testing

\subsection{Common LeetCode Patterns for E-Commerce}

Based on e-commerce/marketplace context, expect questions related to:

\subsubsection{Pattern 1: Arrays \& Hashing}

\textbf{Sample Problem:} Product Inventory Management

\textit{Problem:} Given an array of product IDs representing items in a retailer's cart, find the most frequently purchased product. If there's a tie, return the product with the smallest ID.

\begin{lstlisting}
def most_frequent_product(cart):
    """
    Time: O(n), Space: O(n)
    """
    from collections import Counter

    if not cart:
        return -1

    # Count frequencies
    freq = Counter(cart)

    # Find max frequency
    max_freq = max(freq.values())

    # Among products with max freq, return smallest ID
    candidates = [pid for pid, count in freq.items()
                  if count == max_freq]

    return min(candidates)

# Example
cart = [101, 203, 101, 305, 203, 101, 203]
print(most_frequent_product(cart))  # 101 (freq=3, but smaller ID)
\end{lstlisting}

\textbf{Follow-up:} What if we need to return top K products? Use a heap.

\subsubsection{Pattern 2: Two Pointers}

\textbf{Sample Problem:} Merge Brand Catalogs

\textit{Problem:} Two brands have sorted arrays of product prices. Merge them into one sorted array.

\begin{lstlisting}
def merge_catalogs(prices1, prices2):
    """
    Two pointers to merge sorted arrays.
    Time: O(m + n), Space: O(m + n)
    """
    result = []
    i, j = 0, 0

    while i < len(prices1) and j < len(prices2):
        if prices1[i] <= prices2[j]:
            result.append(prices1[i])
            i += 1
        else:
            result.append(prices2[j])
            j += 1

    # Append remaining
    result.extend(prices1[i:])
    result.extend(prices2[j:])

    return result
\end{lstlisting}

\subsubsection{Pattern 3: Binary Search}

\textbf{Sample Problem:} Find Minimum Price in Rotated Catalog

\textit{Problem:} A sorted price list was rotated. Find the minimum price.

\begin{lstlisting}
def find_min_price(prices):
    """
    Binary search on rotated sorted array.
    Time: O(log n), Space: O(1)
    """
    left, right = 0, len(prices) - 1

    while left < right:
        mid = (left + right) // 2

        # Right half is unsorted, min is in right half
        if prices[mid] > prices[right]:
            left = mid + 1
        else:
            # Left half is unsorted or mid is min
            right = mid

    return prices[left]

# Example
prices = [4.99, 5.99, 6.99, 1.99, 2.99, 3.99]
print(find_min_price(prices))  # 1.99
\end{lstlisting}

\subsubsection{Pattern 4: Trees (Product Taxonomy)}

\textbf{Sample Problem:} Category Tree Traversal

\textit{Problem:} Faire has product categories organized in a tree. Count total products in a category and all subcategories.

\begin{lstlisting}
class CategoryNode:
    def __init__(self, name, product_count):
        self.name = name
        self.product_count = product_count
        self.children = []

def count_products(root):
    """
    DFS to count products in category tree.
    Time: O(n), Space: O(h) for recursion stack
    """
    if not root:
        return 0

    total = root.product_count

    for child in root.children:
        total += count_products(child)

    return total

# Example tree: Home Decor -> (Furniture, Lighting)
root = CategoryNode("Home Decor", 100)
furniture = CategoryNode("Furniture", 50)
lighting = CategoryNode("Lighting", 30)
root.children = [furniture, lighting]

print(count_products(root))  # 180
\end{lstlisting}

\subsubsection{Pattern 5: Graphs (Retailer-Brand Network)}

\textbf{Sample Problem:} Find Connected Retailers

\textit{Problem:} Given a graph where nodes are retailers and edges connect retailers who order from the same brands, find the number of connected components (retailer communities).

\begin{lstlisting}
def count_communities(n, edges):
    """
    Union-Find to count connected components.
    Time: O(n + e * alpha(n)), Space: O(n)
    """
    parent = list(range(n))

    def find(x):
        if parent[x] != x:
            parent[x] = find(parent[x])  # Path compression
        return parent[x]

    def union(x, y):
        root_x, root_y = find(x), find(y)
        if root_x != root_y:
            parent[root_x] = root_y

    # Build connections
    for u, v in edges:
        union(u, v)

    # Count unique roots
    return len(set(find(i) for i in range(n)))

# Example: 5 retailers, edges: [(0,1), (1,2), (3,4)]
print(count_communities(5, [(0,1), (1,2), (3,4)]))  # 2 communities
\end{lstlisting}

\subsubsection{Pattern 6: Dynamic Programming}

\textbf{Sample Problem:} Maximum Order Value (Knapsack Variant)

\textit{Problem:} A retailer has a budget. Given products with prices and profit margins, maximize total profit without exceeding budget.

\begin{lstlisting}
def max_profit(budget, prices, profits):
    """
    0/1 Knapsack problem.
    Time: O(n * budget), Space: O(budget)
    """
    n = len(prices)
    dp = [0] * (budget + 1)

    for i in range(n):
        price = prices[i]
        profit = profits[i]

        # Traverse backwards to avoid using same item twice
        for b in range(budget, price - 1, -1):
            dp[b] = max(dp[b], dp[b - price] + profit)

    return dp[budget]

# Example
budget = 100
prices = [30, 40, 50, 60]
profits = [10, 20, 25, 30]
print(max_profit(budget, prices, profits))  # 50 (items 1 and 2)
\end{lstlisting}

\subsection{E-commerce Specific Coding Questions}

These are deduced based on Faire's product:

\subsubsection{Question 1: Product Search Ranking}

\textit{Problem:} Given a search query and a list of products (each with name, description, price), return the top K most relevant products. Relevance is based on keyword matches (more matches = higher relevance). If tied, prioritize lower price.

\begin{lstlisting}
def search_products(query, products, k):
    """
    Time: O(n * m + n log n), where n = products, m = avg words
    Space: O(n)
    """
    query_words = set(query.lower().split())

    scored_products = []
    for product in products:
        text = (product['name'] + ' ' + product['description']).lower()
        words = set(text.split())

        # Count keyword matches
        matches = len(query_words & words)

        # Score: (matches, -price) for sorting
        scored_products.append((matches, -product['price'], product))

    # Sort by matches (desc), then price (asc)
    scored_products.sort(key=lambda x: (x[0], -x[1]), reverse=True)

    return [p[2] for p in scored_products[:k]]

# Example
products = [
    {"name": "Vase", "description": "ceramic blue vase", "price": 25},
    {"name": "Blue Pot", "description": "ceramic pot", "price": 20},
    {"name": "Ceramic Bowl", "description": "handmade ceramic", "price": 30}
]
query = "blue ceramic"
print(search_products(query, products, 2))
# Returns: Blue Pot (2 matches, $20), Vase (2 matches, $25)
\end{lstlisting}

\subsubsection{Question 2: Order Fulfillment}

\textit{Problem:} Faire processes orders from multiple retailers. Given an array of order timestamps (Unix time), find the maximum number of orders processed in any 1-hour window.

\begin{lstlisting}
def max_orders_in_hour(timestamps):
    """
    Sliding window approach.
    Time: O(n log n + n), Space: O(1)
    """
    if not timestamps:
        return 0

    timestamps.sort()
    max_count = 0

    for i in range(len(timestamps)):
        window_end = timestamps[i] + 3600  # 1 hour in seconds

        # Binary search for end of window
        left, right = i, len(timestamps) - 1
        while left < right:
            mid = (left + right + 1) // 2
            if timestamps[mid] <= window_end:
                left = mid
            else:
                right = mid - 1

        count = left - i + 1
        max_count = max(max_count, count)

    return max_count

# Example
timestamps = [1000, 2000, 3000, 4500, 5000, 6000]
print(max_orders_in_hour(timestamps))  # 4 orders
\end{lstlisting}

\subsection{SQL Questions}

Faire emphasizes SQL and data modeling. Expect questions like:

\subsubsection{Question 1: Top Brands by Revenue}

\textit{Problem:} Given tables \texttt{orders(order\_id, brand\_id, amount, date)} and \texttt{brands(brand\_id, name)}, find the top 5 brands by total revenue in the last 30 days.

\begin{lstlisting}[language=SQL]
SELECT
    b.name,
    SUM(o.amount) AS total_revenue
FROM orders o
JOIN brands b ON o.brand_id = b.brand_id
WHERE o.date >= CURRENT_DATE - INTERVAL 30 DAY
GROUP BY b.brand_id, b.name
ORDER BY total_revenue DESC
LIMIT 5;
\end{lstlisting}

\subsubsection{Question 2: Retailer Churn Rate}

\textit{Problem:} Calculate monthly churn rate. Tables: \texttt{retailers(retailer\_id, signup\_date)}, \texttt{orders(order\_id, retailer\_id, order\_date)}.

\textit{Definition:} Churn = retailers who placed orders in month N but not in month N+1.

\begin{lstlisting}[language=SQL]
WITH monthly_active AS (
    SELECT
        retailer_id,
        DATE_TRUNC('month', order_date) AS month
    FROM orders
    GROUP BY retailer_id, month
),
churned AS (
    SELECT
        m1.month,
        m1.retailer_id
    FROM monthly_active m1
    LEFT JOIN monthly_active m2
        ON m1.retailer_id = m2.retailer_id
        AND m2.month = m1.month + INTERVAL '1 month'
    WHERE m2.retailer_id IS NULL
)
SELECT
    month,
    COUNT(DISTINCT retailer_id) AS churned_retailers
FROM churned
GROUP BY month
ORDER BY month;
\end{lstlisting}

% ============================================================================
\section{System Design - Search Prep (Interview Part 2)}
% ============================================================================

\textbf{Context:} This section covers the Systems Design - Search interview (Part 2 of the interview), which is a 1-hour system design discussion focused on e-commerce search architecture.

\textbf{Key Guidance from Recruiter:}
\begin{itemize}
    \item Focus on major components of e-commerce search system
    \item NOT about super-detailed ML knowledge
    \item Emphasize: data flows, architecture, scaling, tradeoffs
\end{itemize}

\subsection{Why Search is Critical for Faire}

Search is the main system design focus. This aligns with Faire's core value proposition:

\begin{itemize}
    \item \textbf{700K+ retailers} need to discover products from \textbf{100K+ brands}
    \item \textbf{Millions of SKUs} across diverse categories (home decor, apparel, food, etc.)
    \item \textbf{Two shopping modes:}
    \begin{itemize}
        \item \textit{Focused search:} "I need ceramic mugs for my coffee shop"
        \item \textit{Inspirational browsing:} "Show me trending home decor"
    \end{itemize}
\end{itemize}

\subsection{System Design Problem 1: Design Faire's Product Search}

\subsubsection{Problem Statement}

\textit{Design a search system for Faire's wholesale marketplace. Retailers should be able to search for products across millions of SKUs from 100K+ brands. The system should support:}

\begin{itemize}
    \item Keyword search with autocomplete
    \item Filters (price, category, brand, location, minimum order, etc.)
    \item Relevance ranking
    \item Low latency (< 200ms p99)
    \item Personalization based on retailer's store type and past orders
    \item Handle 10,000 QPS at peak
\end{itemize}

\subsubsection{Clarifying Questions}

\textbf{Ask the interviewer:}

\begin{enumerate}
    \item \textbf{Scale:} How many products? How many searches/day?
    \item \textbf{Latency:} What's acceptable p50, p95, p99?
    \item \textbf{Ranking:} Should we optimize for clicks? Add-to-cart? Orders?
    \item \textbf{Personalization:} How much personalization vs. global relevance?
    \item \textbf{Consistency:} Strong consistency needed or eventual consistency OK?
    \item \textbf{Internationalization:} Multiple languages? Multiple currencies?
\end{enumerate}

\subsubsection{Requirements \& Scale Estimation}

\textbf{Functional Requirements:}
\begin{itemize}
    \item Search products by keywords
    \item Filter by attributes (price, category, brand, etc.)
    \item Autocomplete search queries
    \item Rank results by relevance
    \item Personalize results per retailer
\end{itemize}

\textbf{Non-Functional Requirements:}
\begin{itemize}
    \item Low latency: < 200ms p99
    \item High availability: 99.9\% uptime
    \item Scalability: 10K QPS peak
    \item Consistency: Eventual consistency OK for search index
\end{itemize}

\textbf{Scale Estimation:}
\begin{itemize}
    \item \textbf{Products:} 10M SKUs
    \item \textbf{Brands:} 100K brands
    \item \textbf{Retailers:} 700K active retailers
    \item \textbf{Searches:} 1M searches/day = 12 QPS average, 120 QPS peak (10x)
    \item \textbf{Index size:} Assume 10KB per product $\rightarrow$ 100GB index
\end{itemize}

\subsubsection{High-Level Architecture}

\textbf{Components:}

\begin{enumerate}
    \item \textbf{API Gateway:} Load balancing, rate limiting, authentication
    \item \textbf{Search Service:} Handles search requests, queries ElasticSearch
    \item \textbf{ElasticSearch Cluster:} Inverted index for fast keyword search
    \item \textbf{Ranking Service:} ML-based ranking (XGBoost)
    \item \textbf{Feature Store (Redis):} Real-time features for ranking
    \item \textbf{Autocomplete Service:} Trie-based prefix matching
    \item \textbf{Personalization Service:} Retriever embeddings, user preferences
    \item \textbf{Data Ingestion Pipeline:} Update search index from product catalog
\end{enumerate}

\textbf{Data Flow:}

\begin{verbatim}
Retailer -> API Gateway -> Search Service
                              |
                              v
                         ElasticSearch (retrieve top 1000)
                              |
                              v
                         Ranking Service (+ Redis features)
                              |
                              v
                         XGBoost Model (re-rank to top 50)
                              |
                              v
                         Personalization Layer
                              |
                              v
                         Return ranked results
\end{verbatim}

\subsubsection{Detailed Component Design}

\textbf{1. ElasticSearch Cluster}

\textbf{Index Schema:}
\begin{lstlisting}[language=json]
{
  "product_id": "12345",
  "name": "Ceramic Mug Set",
  "description": "Handmade ceramic mugs, set of 4",
  "category": "Home & Kitchen > Drinkware",
  "brand_id": "brand_789",
  "brand_name": "ArtisanCraft Co",
  "price": 45.99,
  "tags": ["ceramic", "handmade", "eco-friendly"],
  "min_order_quantity": 6,
  "inventory_count": 150,
  "brand_location": "Portland, OR",
  "created_at": "2024-01-15T10:00:00Z",
  "popularity_score": 8.5
}
\end{lstlisting}

\textbf{Scoring Function:}

ElasticSearch uses BM25 for text relevance, but we can customize:

\begin{equation}
\text{score} = \alpha \cdot \text{BM25}(q, d) + \beta \cdot \text{popularity}(d) + \gamma \cdot \text{recency}(d)
\end{equation}

Where:
\begin{itemize}
    \item $\text{BM25}(q, d)$: keyword match score
    \item $\text{popularity}(d)$: click-through rate, order rate
    \item $\text{recency}(d)$: boost newer products
    \item $\alpha, \beta, \gamma$: learned weights
\end{itemize}

\textbf{Sharding Strategy:}
\begin{itemize}
    \item Shard by product ID hash (uniform distribution)
    \item 10 shards for 10M products = 1M products/shard
    \item 3 replicas per shard for high availability
\end{itemize}

\textbf{2. Ranking Service with XGBoost}

After ElasticSearch retrieves 1,000 candidates, re-rank with ML model.

\textbf{Features (hundreds per product):}
\begin{itemize}
    \item \textbf{Text relevance:} BM25 score, title match, description match
    \item \textbf{Retailer-product similarity:} Embedding cosine similarity
    \item \textbf{Engagement signals:} CTR, add-to-cart rate, purchase rate
    \item \textbf{Product attributes:} Price, category, brand popularity, inventory
    \item \textbf{Personalization:} Retailer's past category preferences, price sensitivity
\end{itemize}

\textbf{Model Training:}
\begin{itemize}
    \item \textbf{Objective:} Predict probability of order given search query
    \item \textbf{Training data:} Search queries + clicked/ordered products (positive), random products (negative)
    \item \textbf{Model:} XGBoost (gradient boosted trees)
    \item \textbf{Offline evaluation:} NDCG@10, Precision@10, Recall@50
    \item \textbf{Online evaluation:} A/B testing (CTR, conversion rate, revenue)
\end{itemize}

\textbf{Feature Retrieval:}
\begin{itemize}
    \item Store features in Redis (key: product\_id, value: feature vector)
    \item Batch fetch 1,000 products' features in parallel
    \item Latency: < 10ms for Redis lookup
\end{itemize}

\textbf{3. Autocomplete Service}

\textbf{Data Structure: Trie (Prefix Tree)}

\begin{lstlisting}
class TrieNode:
    def __init__(self):
        self.children = {}
        self.is_end_of_word = False
        self.suggestions = []  # Top 5 completions

class Autocomplete:
    def __init__(self):
        self.root = TrieNode()

    def insert(self, query, frequency):
        node = self.root
        for char in query.lower():
            if char not in node.children:
                node.children[char] = TrieNode()
            node = node.children[char]

        node.is_end_of_word = True
        # Update top suggestions at this node
        self.update_suggestions(node, query, frequency)

    def search(self, prefix):
        node = self.root
        for char in prefix.lower():
            if char not in node.children:
                return []
            node = node.children[char]

        return node.suggestions[:5]
\end{lstlisting}

\textbf{Populating Autocomplete:}
\begin{itemize}
    \item Mine frequent queries from search logs
    \item Use product names, brand names, category names
    \item Update daily from offline job
    \item Store in Redis for low-latency lookup
\end{itemize}

\textbf{4. Personalization Layer}

Use \textbf{embedding-based retrieval} to personalize results.

\textbf{Embeddings:}
\begin{itemize}
    \item \textbf{Retailer embeddings:} Learned from order history, clicks, store category
    \item \textbf{Product embeddings:} Learned from co-purchase patterns, attributes
    \item \textbf{Similarity:} Cosine similarity in embedding space
\end{itemize}

\textbf{Model:} Two-tower neural network (similar to YouTube's DNN)
\begin{itemize}
    \item \textbf{Retailer tower:} Encode retailer features $\rightarrow$ retailer embedding
    \item \textbf{Product tower:} Encode product features $\rightarrow$ product embedding
    \item \textbf{Training:} Maximize dot product for positive pairs (retailer ordered product)
\end{itemize}

\textbf{Personalization at Serving:}
\begin{lstlisting}
def personalize(ranked_products, retailer_embedding):
    """
    Re-weight top 50 products by retailer-product similarity.
    """
    for product in ranked_products:
        similarity = cosine_similarity(
            retailer_embedding,
            product.embedding
        )
        product.score = 0.7 * product.score + 0.3 * similarity

    return sorted(ranked_products, key=lambda p: p.score, reverse=True)
\end{lstlisting}

\subsubsection{Database Schema}

\textbf{Products Table:}
\begin{lstlisting}[language=SQL]
CREATE TABLE products (
    product_id BIGINT PRIMARY KEY,
    brand_id BIGINT,
    name VARCHAR(255),
    description TEXT,
    category_id INT,
    price DECIMAL(10, 2),
    min_order_qty INT,
    inventory_count INT,
    created_at TIMESTAMP,
    updated_at TIMESTAMP,
    INDEX (brand_id),
    INDEX (category_id)
);
\end{lstlisting}

\textbf{Search Logs Table:}
\begin{lstlisting}[language=SQL]
CREATE TABLE search_logs (
    log_id BIGINT PRIMARY KEY AUTO_INCREMENT,
    retailer_id BIGINT,
    query VARCHAR(500),
    clicked_product_id BIGINT,
    position INT,
    timestamp TIMESTAMP,
    INDEX (retailer_id, timestamp),
    INDEX (query)
);
\end{lstlisting}

\subsubsection{API Design}

\textbf{Search API:}
\begin{lstlisting}[language=json]
GET /api/v1/search?q=ceramic+mugs&filters={"category":"drinkware","price_max":50}&page=1&size=20

Response:
{
  "query": "ceramic mugs",
  "total_results": 1523,
  "page": 1,
  "size": 20,
  "products": [
    {
      "product_id": 12345,
      "name": "Ceramic Mug Set",
      "brand": "ArtisanCraft Co",
      "price": 45.99,
      "image_url": "https://...",
      "min_order": 6
    },
    ...
  ],
  "facets": {
    "categories": {"drinkware": 800, "home_decor": 300},
    "price_ranges": {"0-25": 500, "25-50": 700, "50-100": 323}
  }
}
\end{lstlisting}

\textbf{Autocomplete API:}
\begin{lstlisting}[language=json]
GET /api/v1/autocomplete?q=cera

Response:
{
  "suggestions": [
    "ceramic mugs",
    "ceramic vases",
    "ceramic plates",
    "ceramic bowls",
    "ceramic decor"
  ]
}
\end{lstlisting}

\subsubsection{Optimizations \& Tradeoffs}

\textbf{Caching Strategy:}
\begin{itemize}
    \item \textbf{CDN:} Cache popular search results (top 1000 queries)
    \item \textbf{Redis:} Cache recent searches per retailer
    \item \textbf{TTL:} 5 minutes for search results (balance freshness vs. performance)
\end{itemize}

\textbf{Cold Start Problem:}
\begin{itemize}
    \item \textbf{New retailers:} Use store category to initialize preferences
    \item \textbf{New products:} Boost in ranking for visibility, collect data quickly
\end{itemize}

\textbf{Handling Spikes:}
\begin{itemize}
    \item Auto-scaling ElasticSearch cluster
    \item Rate limiting per retailer (prevent abuse)
    \item Degrade gracefully: skip personalization if Redis is slow
\end{itemize}

\textbf{Consistency Tradeoffs:}
\begin{itemize}
    \item Product updates: eventual consistency OK (few minutes delay)
    \item Critical updates (out-of-stock): real-time update via pub/sub
\end{itemize}

\subsubsection{Metrics \& Monitoring}

\textbf{System Metrics:}
\begin{itemize}
    \item Latency: p50, p95, p99
    \item QPS, error rate, cache hit rate
    \item ElasticSearch shard health
\end{itemize}

\textbf{Business Metrics:}
\begin{itemize}
    \item CTR (click-through rate) per search
    \item Conversion rate (search $\rightarrow$ order)
    \item Revenue per search
    \item Zero-result rate (searches with no results)
\end{itemize}

\textbf{ML Metrics:}
\begin{itemize}
    \item NDCG@10 (offline)
    \item Precision@5, Recall@20 (offline)
    \item Online A/B testing (CTR, conversion)
\end{itemize}

\subsection{System Design Problem 2: Design Faire's Recommendation System}

\subsubsection{Problem Statement}

\textit{Design a recommendation system for Faire's homepage and product pages. When a retailer logs in, show personalized product recommendations. Also show "You may also like" on product detail pages.}

\textbf{Requirements:}
\begin{itemize}
    \item Personalized homepage recommendations (50 products)
    \item "Similar products" on product pages (10 products)
    \item Handle cold start (new retailers, new products)
    \item Low latency (< 100ms)
    \item Optimize for order conversion, not just clicks
\end{itemize}

\subsubsection{Approach: Two-Stage Retrieval + Ranking}

This mirrors Faire's actual architecture (based on their engineering blog).

\textbf{Stage 1: Candidate Generation}
\begin{itemize}
    \item Retrieve 1,000 candidate products from multiple sources:
    \begin{enumerate}
        \item \textbf{Collaborative filtering:} Retailers similar to you ordered these
        \item \textbf{Content-based:} Products similar to what you've ordered
        \item \textbf{Trending:} Popular products in your category
        \item \textbf{New arrivals:} Recently added products
    \end{enumerate}
\end{itemize}

\textbf{Stage 2: Ranking}
\begin{itemize}
    \item Score 1,000 candidates with XGBoost
    \item Features: retailer-product similarity, engagement, diversity
    \item Return top 50
\end{itemize}

\subsubsection{Collaborative Filtering with Embeddings}

Faire uses \textbf{embedding-based CF} learned from implicit feedback.

\textbf{Training Data:}
\begin{itemize}
    \item \textbf{Positive:} (retailer, product) pairs from orders, add-to-cart, favorites
    \item \textbf{Negative:} Random products (or hard negatives: clicked but not ordered)
\end{itemize}

\textbf{Model: Two-Tower DNN}

\begin{lstlisting}
import torch
import torch.nn as nn

class TwoTowerModel(nn.Module):
    def __init__(self, num_retailers, num_products, embedding_dim=128):
        super().__init__()

        # Retailer tower
        self.retailer_embedding = nn.Embedding(num_retailers, embedding_dim)
        self.retailer_fc = nn.Sequential(
            nn.Linear(embedding_dim, 256),
            nn.ReLU(),
            nn.Linear(256, 128)
        )

        # Product tower
        self.product_embedding = nn.Embedding(num_products, embedding_dim)
        self.product_fc = nn.Sequential(
            nn.Linear(embedding_dim, 256),
            nn.ReLU(),
            nn.Linear(256, 128)
        )

    def forward(self, retailer_ids, product_ids):
        # Encode retailer
        retailer_emb = self.retailer_embedding(retailer_ids)
        retailer_vec = self.retailer_fc(retailer_emb)

        # Encode product
        product_emb = self.product_embedding(product_ids)
        product_vec = self.product_fc(product_emb)

        # Dot product (similarity)
        scores = (retailer_vec * product_vec).sum(dim=1)

        return torch.sigmoid(scores)
\end{lstlisting}

\textbf{Loss Function:}
\begin{equation}
\mathcal{L} = -\sum_{(r,p) \in \text{positive}} \log \sigma(e_r \cdot e_p) - \sum_{(r,p') \in \text{negative}} \log(1 - \sigma(e_r \cdot e_{p'}))
\end{equation}

\textbf{Serving:}
\begin{itemize}
    \item Store product embeddings in FAISS (Approximate Nearest Neighbors)
    \item Given retailer embedding, find top K nearest products
    \item Latency: < 10ms for ANN search
\end{itemize}

\subsubsection{Product-to-Product Recommendations}

For "You may also like" on product pages:

\textbf{Approach 1: Co-Purchase Matrix}
\begin{itemize}
    \item Build matrix: product A $\times$ product B = count of co-purchases
    \item Use collaborative filtering (item-item similarity)
    \item Cosine similarity in co-purchase space
\end{itemize}

\textbf{Approach 2: Embedding Similarity}
\begin{itemize}
    \item Use product embeddings from two-tower model
    \item For product P, find K nearest neighbors in embedding space
    \item Pre-compute and cache in Redis
\end{itemize}

\subsubsection{Diversity \& Exploration}

Avoid showing only similar products (filter bubble).

\textbf{Diversity Technique: MMR (Maximal Marginal Relevance)}

\begin{equation}
\text{MMR}(p) = \lambda \cdot \text{relevance}(p) - (1 - \lambda) \cdot \max_{p' \in S} \text{similarity}(p, p')
\end{equation}

Where:
\begin{itemize}
    \item $S$: already selected products
    \item $\lambda$: tradeoff between relevance and diversity (e.g., 0.7)
\end{itemize}

\textbf{Exploration: Thompson Sampling}
\begin{itemize}
    \item Occasionally show new/underexplored products
    \item Use multi-armed bandit to balance exploration vs. exploitation
\end{itemize}

\subsubsection{Two-Sided Optimization}

Faire must balance:
\begin{itemize}
    \item \textbf{Retailer utility:} Relevant, high-quality products
    \item \textbf{Brand utility:} Sufficient exposure, especially for new brands
\end{itemize}

\textbf{Approach:}
\begin{itemize}
    \item Add \textbf{brand diversity} constraint: limit products per brand
    \item Boost \textbf{new brands} in ranking (time decay)
    \item Monitor brand-level metrics: impression share, order share
\end{itemize}

\subsection{Follow-Up Questions}

Be prepared for these follow-ups:

\begin{enumerate}
    \item \textbf{How would you handle multi-language search?}
    \begin{itemize}
        \item Use language-specific analyzers in ElasticSearch
        \item Cross-lingual embeddings for semantic search
        \item Translate queries to English, then search
    \end{itemize}

    \item \textbf{How to prevent spam/fake products in search results?}
    \begin{itemize}
        \item ML classifier to detect spam (based on title, description patterns)
        \item Human review queue for flagged products
        \item Downrank products with low engagement or high return rate
    \end{itemize}

    \item \textbf{How to A/B test ranking algorithms?}
    \begin{itemize}
        \item Split traffic by retailer ID hash
        \item Track metrics: CTR, conversion, revenue per search
        \item Use statistical significance testing (t-test, chi-square)
        \item Gradual rollout: 5\% $\rightarrow$ 10\% $\rightarrow$ 50\% $\rightarrow$ 100\%
    \end{itemize}

    \item \textbf{What if ElasticSearch goes down?}
    \begin{itemize}
        \item Fallback to cached popular queries
        \item Degrade to simpler search (MySQL full-text search)
        \item Show trending products as fallback
        \item Alert on-call engineer, auto-restart nodes
    \end{itemize}

    \item \textbf{How to scale to 100K QPS?}
    \begin{itemize}
        \item Horizontal scaling: more ElasticSearch nodes
        \item Read replicas for feature store (Redis)
        \item CDN for static content and popular searches
        \item Database sharding by geography or retailer segment
    \end{itemize}
\end{enumerate}

% ============================================================================
\section{Algorithm Interview Prep (Interview Part 3)}
% ============================================================================

\textbf{CRITICAL:} This is \textbf{NOT} a LeetCode coding interview! Read this section carefully.

\subsection{What Makes This Different}

\textbf{From Official Recruiter Document:}
\begin{itemize}
    \item \textbf{NO CODE REQUIRED} - This is a discussion, not a coding exercise
    \item \textbf{Format:} 1-2 questions with sub-questions
    \item \textbf{Goal:} Evaluate practical design knowledge for ML-powered systems
    \item \textbf{Focus:} Improve EXISTING systems, not design from scratch
    \item \textbf{Skills Tested:}
    \begin{itemize}
        \item Identify performance bottlenecks
        \item Propose incremental improvements
        \item Suggest architectural changes
        \item Think critically about ML system tradeoffs
    \end{itemize}
\end{itemize}

\subsection{Example Question Format}

\textbf{Scenario:} "Here's our current search ranking system. It retrieves 1000 candidates from ElasticSearch, then re-ranks with XGBoost using 200 features from Redis. The system has the following issues:
\begin{itemize}
    \item P95 latency is 800ms (target: 300ms)
    \item Cold-start retailers get poor recommendations
    \item Feature drift - model trained 3 months ago, performance degrading
\end{itemize}

\textbf{Questions:}
\begin{enumerate}
    \item What would you investigate first to reduce latency?
    \item How would you approach the cold-start problem?
    \item What's your strategy for detecting and handling feature drift?
    \item What incremental changes would you make vs. major architectural changes?
\end{enumerate}

\subsection{How to Approach This Interview}

\subsubsection{Step 1: Understand the Existing System}

Ask clarifying questions:
\begin{itemize}
    \item What are the main components? (retrieval, ranking, serving)
    \item What's the current data flow?
    \item What metrics are being optimized? (latency, relevance, etc.)
    \item What's the scale? (QPS, data volume, model size)
    \item What infrastructure is in place? (caching, load balancing, etc.)
\end{itemize}

\subsubsection{Step 2: Identify Bottlenecks}

Think systematically:
\begin{itemize}
    \item \textbf{Data bottlenecks:} Slow feature retrieval, database queries
    \item \textbf{Compute bottlenecks:} Model inference time, CPU/GPU utilization
    \item \textbf{Network bottlenecks:} Inter-service latency, data transfer
    \item \textbf{ML-specific:} Model size, feature dimensionality, batch size
\end{itemize}

\subsubsection{Step 3: Propose Solutions (Incremental + Architectural)}

\textbf{Incremental Improvements} (quick wins):
\begin{itemize}
    \item Add caching layers (Redis for features, CDN for results)
    \item Optimize database queries (indexes, denormalization)
    \item Feature selection (reduce 200 features to 50 most important)
    \item Model compression (quantization, pruning)
    \item Batch processing where possible
\end{itemize}

\textbf{Architectural Changes} (longer-term):
\begin{itemize}
    \item Move from synchronous to asynchronous feature retrieval
    \item Implement multi-stage ranking (fast initial filter, expensive final ranking)
    \item Build feature store for offline/online consistency
    \item Add monitoring and alerting for drift detection
    \item Implement online learning or periodic retraining pipeline
\end{itemize}

\subsection{Practice Problems}

\subsubsection{Problem 1: High Latency in Search Ranking}

\textbf{System:} ElasticSearch retrieves 1000 candidates, XGBoost re-ranks using 300 features from Redis.

\textbf{Issue:} P95 latency 1.2 seconds (target: 400ms)

\textbf{Your task:}
\begin{enumerate}
    \item Where would you look for bottlenecks? (network, compute, data access)
    \item What quick wins could reduce latency by 30\%?
    \item What architectural changes for 3x improvement?
\end{enumerate}

\textbf{Example Answer:}

\textit{Step 1: Identify bottlenecks}
\begin{itemize}
    \item Profile the request: measure time for retrieval vs. ranking vs. feature fetching
    \item Likely culprit: fetching 300 features × 1000 candidates = 300K Redis lookups
    \item Secondary issue: XGBoost inference on 1000 candidates might be slow
\end{itemize}

\textit{Step 2: Quick wins (30\% improvement)}
\begin{itemize}
    \item \textbf{Feature caching:} Cache popular product features (80/20 rule)
    \item \textbf{Batch Redis reads:} Use pipelining to fetch features in parallel
    \item \textbf{Reduce candidates:} Re-rank only top 500 instead of 1000
    \item \textbf{Feature selection:} Use only top 100 features instead of 300
\end{itemize}

\textit{Step 3: Architectural changes (3x improvement)}
\begin{itemize}
    \item \textbf{Multi-stage ranking:}
    \begin{enumerate}
        \item Stage 1: Fast linear model on 20 features (1000 → 200 candidates)
        \item Stage 2: Full XGBoost on 300 features (200 → final ranking)
    \end{enumerate}
    \item \textbf{Feature store:} Pre-compute and batch-update features (hourly instead of real-time)
    \item \textbf{Model optimization:} Quantize XGBoost model (FP32 → FP16 or INT8)
    \item \textbf{Async processing:} Return fast results, refine in background
\end{itemize}

\subsubsection{Problem 2: Cold-Start Recommendations}

\textbf{System:} Collaborative filtering model recommends products based on retailer's purchase history.

\textbf{Issue:} New retailers with 0-2 orders get irrelevant recommendations.

\textbf{Your task:}
\begin{enumerate}
    \item Why does collaborative filtering fail for new users?
    \item What signals can you use when there's no purchase history?
    \item How would you blend multiple strategies?
\end{enumerate}

\textbf{Example Answer:}

\textit{Step 1: Why it fails}
\begin{itemize}
    \item Collaborative filtering needs user-item interactions
    \item New retailers have sparse or zero interaction data
    \item Model can't find similar retailers to learn from
\end{itemize}

\textit{Step 2: Alternative signals}
\begin{itemize}
    \item \textbf{Content-based:} Use retailer profile (category, location, store size)
    \item \textbf{Popularity:} Show trending/best-selling products in retailer's category
    \item \textbf{Geographic:} Recommend products popular in retailer's region
    \item \textbf{Onboarding data:} Ask preferences during signup (style, price range)
\end{itemize}

\textit{Step 3: Blended strategy}
\begin{itemize}
    \item Use a \textbf{multi-armed bandit} or \textbf{ensemble model}
    \item Weight strategies by data availability:
    \begin{itemize}
        \item 0-5 orders: 70\% content + 30\% popularity
        \item 5-20 orders: 50\% content + 30\% collaborative + 20\% popularity
        \item 20+ orders: 80\% collaborative + 20\% content
    \end{itemize}
    \item \textbf{Exploration:} Inject 10-20\% random/diverse items to learn preferences faster
\end{itemize}

\subsubsection{Problem 3: Feature Drift Detection}

\textbf{System:} XGBoost ranking model trained 6 months ago, performance degrading.

\textbf{Issue:} Click-through rate dropped from 12\% to 9\% over 3 months.

\textbf{Your task:}
\begin{enumerate}
    \item What causes feature drift?
    \item How would you detect it?
    \item What's your retraining strategy?
\end{enumerate}

\textbf{Example Answer:}

\textit{Step 1: Causes of drift}
\begin{itemize}
    \item \textbf{Data distribution shift:} New products, seasonal changes, user behavior changes
    \item \textbf{Feature distribution shift:} Price ranges changed, new brands added
    \item \textbf{Concept drift:} Relationship between features and labels changed
    \item Example: During COVID, "home office" products surged
\end{itemize}

\textit{Step 2: Detection strategies}
\begin{itemize}
    \item \textbf{Monitor feature distributions:} Compare train vs. live (KL divergence, PSI)
    \item \textbf{Track model performance:} Daily CTR, NDCG, precision@K
    \item \textbf{A/B test against random:} If model barely beats random, it's drifted
    \item \textbf{Shadow models:} Train weekly model, compare performance to production
\end{itemize}

\textit{Step 3: Retraining strategy}
\begin{itemize}
    \item \textbf{Schedule:} Retrain monthly (or when drift detected)
    \item \textbf{Data:} Use last 3-6 months of data (balance recency vs. volume)
    \item \textbf{Validation:} A/B test new model before full rollout
    \item \textbf{Gradual rollout:} 5\% → 25\% → 50\% → 100\% traffic
    \item \textbf{Rollback plan:} Keep previous model ready for instant rollback
    \item \textbf{Online learning:} Consider incremental updates between full retrains
\end{itemize}

\subsection{Key Concepts to Know}

\subsubsection{ML System Performance Issues}
\begin{itemize}
    \item \textbf{Latency:} Model inference time, feature retrieval, batching
    \item \textbf{Throughput:} QPS limits, model serving capacity
    \item \textbf{Cost:} Compute costs (GPU), storage costs (feature store)
    \item \textbf{Accuracy:} Drift, cold-start, data quality
\end{itemize}

\subsubsection{Common Optimization Techniques}
\begin{itemize}
    \item \textbf{Caching:} Feature caching, result caching, model caching
    \item \textbf{Model compression:} Quantization, pruning, distillation
    \item \textbf{Feature selection:} Reduce dimensionality, remove correlated features
    \item \textbf{Multi-stage ranking:} Fast filter + expensive rerank
    \item \textbf{Asynchronous processing:} Return fast results, refine later
\end{itemize}

\subsubsection{System Design Patterns for ML}
\begin{itemize}
    \item \textbf{Feature store:} Centralized feature management (offline + online)
    \item \textbf{Model registry:} Versioning, A/B testing, rollback
    \item \textbf{Monitoring:} Drift detection, performance tracking, alerting
    \item \textbf{Feedback loops:} Log predictions, retrain with outcomes
\end{itemize}

\subsection{How to Prepare}

\begin{enumerate}
    \item \textbf{Study Faire's architecture:} Read their engineering blog (craft.faire.com)
    \item \textbf{Practice system analysis:} Take existing ML systems (Netflix, Spotify), identify bottlenecks
    \item \textbf{Learn optimization techniques:} Caching, model compression, multi-stage ranking
    \item \textbf{Understand tradeoffs:} Latency vs. accuracy, cost vs. performance, complexity vs. maintainability
    \item \textbf{Think incrementally:} Always propose quick wins before major rewrites
\end{enumerate}

% ============================================================================
\section{Machine Learning Evaluation Metrics (For All Interviews)}
% ============================================================================

\subsection{Why ML Metrics Matter for Faire}

Faire's search and recommendations are ML-powered. You need to:
\begin{itemize}
    \item Understand common ML evaluation metrics
    \item Know when to use each metric
    \item Articulate tradeoffs (precision vs. recall, etc.)
    \item Relate metrics to business goals
\end{itemize}

\subsection{Classification Metrics}

\subsubsection{Confusion Matrix}

\begin{center}
\begin{tabular}{|c|c|c|}
\hline
& \textbf{Predicted Positive} & \textbf{Predicted Negative} \\
\hline
\textbf{Actual Positive} & True Positive (TP) & False Negative (FN) \\
\hline
\textbf{Actual Negative} & False Positive (FP) & True Negative (TN) \\
\hline
\end{tabular}
\end{center}

\subsubsection{Core Metrics}

\textbf{1. Accuracy}
\begin{equation}
\text{Accuracy} = \frac{TP + TN}{TP + TN + FP + FN}
\end{equation}

\textbf{When to use:} Balanced classes

\textbf{Limitation:} Misleading for imbalanced datasets (e.g., fraud detection: 99\% non-fraud)

\textbf{2. Precision}
\begin{equation}
\text{Precision} = \frac{TP}{TP + FP}
\end{equation}

\textbf{Interpretation:} Of all predicted positives, how many are correct?

\textbf{When to use:} Cost of FP is high (e.g., spam detection - don't mark legitimate emails as spam)

\textbf{Faire Example:} Predicting "retailer will order this product"
\begin{itemize}
    \item High precision: most recommended products are relevant
    \item Low precision: many irrelevant products shown
\end{itemize}

\textbf{3. Recall (Sensitivity, True Positive Rate)}
\begin{equation}
\text{Recall} = \frac{TP}{TP + FN}
\end{equation}

\textbf{Interpretation:} Of all actual positives, how many did we catch?

\textbf{When to use:} Cost of FN is high (e.g., disease detection - don't miss sick patients)

\textbf{Faire Example:} Search results
\begin{itemize}
    \item High recall: retrieve all relevant products
    \item Low recall: miss many relevant products
\end{itemize}

\textbf{4. F1 Score}
\begin{equation}
F1 = 2 \cdot \frac{\text{Precision} \times \text{Recall}}{\text{Precision} + \text{Recall}} = \frac{2TP}{2TP + FP + FN}
\end{equation}

\textbf{When to use:} Balance between precision and recall, imbalanced classes

\textbf{Limitation:} Treats precision and recall equally (sometimes one matters more)

\textbf{5. ROC-AUC}

\textbf{ROC Curve:} Plot of True Positive Rate (Recall) vs. False Positive Rate at various thresholds

\begin{equation}
\text{FPR} = \frac{FP}{FP + TN}
\end{equation}

\textbf{AUC (Area Under Curve):} Probability that model ranks random positive higher than random negative

\textbf{When to use:} Compare models, threshold-agnostic evaluation

\textbf{Limitation:} Optimistic for imbalanced datasets (use PR-AUC instead)

\subsubsection{Example Interview Question}

\textit{Question:} Faire wants to predict which products a retailer will order. You build a model with 90\% accuracy. Is this good?

\textbf{Answer:}
\begin{itemize}
    \item \textbf{Need more context:} What's the class distribution?
    \item If only 5\% of shown products are ordered, a naive model (always predict "no order") has 95\% accuracy but is useless.
    \item \textbf{Better metrics:} Precision@K, Recall@K (see ranking metrics below)
    \item \textbf{Business metric:} Revenue per recommended product
\end{itemize}

\subsection{Ranking Metrics}

For search and recommendations, we care about \textbf{order} of results, not just classification.

\subsubsection{Precision@K and Recall@K}

\textbf{Precision@K:}
\begin{equation}
\text{Precision@K} = \frac{\text{\# relevant items in top K}}{K}
\end{equation}

\textbf{Recall@K:}
\begin{equation}
\text{Recall@K} = \frac{\text{\# relevant items in top K}}{\text{total relevant items}}
\end{equation}

\textbf{Example:} Search for "ceramic mugs", 100 total relevant products

Top 10 results: 7 relevant, 3 irrelevant
\begin{itemize}
    \item Precision@10 = 7/10 = 0.7
    \item Recall@10 = 7/100 = 0.07
\end{itemize}

\textbf{Faire Use Case:} Precision@10 for search (top 10 results matter most)

\subsubsection{Mean Average Precision (MAP)}

\textbf{Average Precision (AP):} Average of precision values at each relevant position

\begin{equation}
AP = \frac{1}{|\text{relevant}|} \sum_{k=1}^{n} \text{Precision@k} \times \text{rel}(k)
\end{equation}

Where $\text{rel}(k) = 1$ if item at position $k$ is relevant, else 0.

\textbf{MAP:} Average AP across multiple queries

\textbf{Example:}

Ranked list: [R, NR, R, R, NR, R] (R = relevant, NR = not relevant)

Relevant positions: 1, 3, 4, 6

Precision at each:
\begin{itemize}
    \item P@1 = 1/1 = 1.0
    \item P@3 = 2/3 = 0.67
    \item P@4 = 3/4 = 0.75
    \item P@6 = 4/6 = 0.67
\end{itemize}

$AP = \frac{1}{4}(1.0 + 0.67 + 0.75 + 0.67) = 0.77$

\textbf{Limitation:} Doesn't consider position (item at rank 1 vs. rank 10 weighted equally)

\subsubsection{Normalized Discounted Cumulative Gain (NDCG)}

\textbf{Why NDCG?} Addresses two issues:
\begin{enumerate}
    \item \textbf{Graded relevance:} Items can be highly relevant, somewhat relevant, or not relevant (not just binary)
    \item \textbf{Position matters:} Higher-ranked items should be more relevant
\end{enumerate}

\textbf{Cumulative Gain (CG):}
\begin{equation}
CG@K = \sum_{i=1}^{K} \text{rel}_i
\end{equation}

Where $\text{rel}_i$ is relevance score of item at position $i$ (e.g., 0, 1, 2, 3)

\textbf{Discounted Cumulative Gain (DCG):}
\begin{equation}
DCG@K = \sum_{i=1}^{K} \frac{\text{rel}_i}{\log_2(i + 1)}
\end{equation}

Position discount: items at lower ranks contribute less

\textbf{Ideal DCG (IDCG):}

DCG if items were perfectly ranked by relevance

\textbf{Normalized DCG (NDCG):}
\begin{equation}
NDCG@K = \frac{DCG@K}{IDCG@K}
\end{equation}

Range: [0, 1], where 1 = perfect ranking

\textbf{Example:}

Ranked list with relevance scores: [3, 2, 0, 1, 2]

$DCG@5 = \frac{3}{\log_2 2} + \frac{2}{\log_2 3} + \frac{0}{\log_2 4} + \frac{1}{\log_2 5} + \frac{2}{\log_2 6}$

$DCG@5 = 3.0 + 1.26 + 0 + 0.43 + 0.77 = 5.46$

Ideal ranking: [3, 2, 2, 1, 0]

$IDCG@5 = 3.0 + 1.26 + 0.77 + 0.43 + 0 = 5.46$

$NDCG@5 = 5.46 / 5.46 = 1.0$ (if we had perfect ranking)

For actual ranking: $NDCG@5 = 5.46 / 5.46 = 1.0$ (happens to be perfect in this case)

\textbf{Faire Use Case:} NDCG@10 for evaluating search ranking offline

\subsubsection{Hit Rate (Recall@K for Recommendations)}

\textbf{Definition:} Proportion of users who found at least one relevant item in top K recommendations

\begin{equation}
\text{Hit Rate@K} = \frac{\text{\# users with } \geq 1 \text{ relevant item in top K}}{\text{total users}}
\end{equation}

\textbf{Faire Example:}

100 retailers, show top 10 recommendations
\begin{itemize}
    \item 80 retailers ordered at least one recommended product
    \item Hit Rate@10 = 80/100 = 0.8
\end{itemize}

\subsection{Regression Metrics}

If predicting continuous values (e.g., estimated order value):

\subsubsection{Mean Absolute Error (MAE)}

\begin{equation}
MAE = \frac{1}{n} \sum_{i=1}^{n} |y_i - \hat{y}_i|
\end{equation}

\textbf{Pros:} Easy to interpret (average error in same units as target)

\textbf{Cons:} Treats all errors equally (large errors not penalized more)

\subsubsection{Mean Squared Error (MSE)}

\begin{equation}
MSE = \frac{1}{n} \sum_{i=1}^{n} (y_i - \hat{y}_i)^2
\end{equation}

\textbf{Pros:} Penalizes large errors more heavily

\textbf{Cons:} Units are squared (harder to interpret)

\subsubsection{Root Mean Squared Error (RMSE)}

\begin{equation}
RMSE = \sqrt{\frac{1}{n} \sum_{i=1}^{n} (y_i - \hat{y}_i)^2}
\end{equation}

\textbf{Pros:} Same units as target, penalizes large errors

\subsubsection{R-Squared ($R^2$)}

\begin{equation}
R^2 = 1 - \frac{\sum_{i=1}^{n} (y_i - \hat{y}_i)^2}{\sum_{i=1}^{n} (y_i - \bar{y})^2}
\end{equation}

\textbf{Interpretation:} Proportion of variance explained by model

\textbf{Range:} $(-\infty, 1]$, where 1 = perfect predictions, 0 = model is as good as mean baseline

\subsection{A/B Testing \& Online Metrics}

Offline metrics (NDCG, Precision@K) don't always correlate with business outcomes. Must A/B test.

\subsubsection{Online Metrics for Search/Recommendations}

\textbf{Engagement Metrics:}
\begin{itemize}
    \item \textbf{Click-Through Rate (CTR):} clicks / impressions
    \item \textbf{Add-to-Cart Rate:} adds / impressions
    \item \textbf{Conversion Rate:} orders / impressions
\end{itemize}

\textbf{Business Metrics:}
\begin{itemize}
    \item \textbf{Revenue per search}
    \item \textbf{Average order value (AOV)}
    \item \textbf{Customer lifetime value (LTV)}
\end{itemize}

\textbf{User Experience:}
\begin{itemize}
    \item \textbf{Session duration}
    \item \textbf{Bounce rate}
    \item \textbf{Zero-result rate} (searches with no results)
\end{itemize}

\subsubsection{A/B Test Design}

\textbf{Hypothesis:} New ranking model increases conversion rate

\textbf{Setup:}
\begin{itemize}
    \item \textbf{Control group:} 50\% of retailers see current ranking
    \item \textbf{Treatment group:} 50\% see new ranking
    \item \textbf{Randomization:} Hash retailer ID to assign group
\end{itemize}

\textbf{Duration:} Run for 2 weeks (capture weekly patterns)

\textbf{Sample Size:} Calculate needed sample for statistical power

\begin{equation}
n = \frac{2(Z_{\alpha/2} + Z_\beta)^2 \sigma^2}{(\mu_1 - \mu_0)^2}
\end{equation}

\textbf{Statistical Test:}
\begin{itemize}
    \item \textbf{Metric:} Conversion rate (proportion)
    \item \textbf{Test:} Two-proportion z-test
    \item \textbf{Significance level:} $\alpha = 0.05$
\end{itemize}

\textbf{Decision:}
\begin{itemize}
    \item If $p < 0.05$ and treatment is better: ship new model
    \item If not significant: keep investigating (longer test, larger sample)
    \item Monitor for novelty effect (users like new things initially)
\end{itemize}

\subsection{Common ML Interview Questions}

\subsubsection{Question 1: Precision vs. Recall Tradeoff}

\textit{You're building a spam product detector for Faire. Would you optimize for precision or recall? Why?}

\textbf{Answer:}

\textbf{Optimize for precision} (minimize false positives).

\textbf{Reasoning:}
\begin{itemize}
    \item \textbf{FP (false positive):} Legitimate product marked as spam
    \begin{itemize}
        \item Harm: Brand loses sales, may leave platform
        \item High cost to business
    \end{itemize}
    \item \textbf{FN (false negative):} Spam product not caught
    \begin{itemize}
        \item Harm: Some low-quality products shown
        \item Lower cost (can be caught by user reports)
    \end{itemize}
\end{itemize}

\textbf{Approach:}
\begin{itemize}
    \item Set high threshold for spam classification (e.g., 0.95 probability)
    \item Human review queue for borderline cases
    \item Monitor precision/recall in production
\end{itemize}

\subsubsection{Question 2: Imbalanced Dataset}

\textit{Only 2\% of shown products are actually ordered. How do you handle this in training?}

\textbf{Answer:}

\begin{enumerate}
    \item \textbf{Resampling:}
    \begin{itemize}
        \item \textbf{Oversample} minority class (ordered products)
        \item \textbf{Undersample} majority class (not ordered)
        \item Use SMOTE (Synthetic Minority Over-sampling)
    \end{itemize}

    \item \textbf{Class Weights:}
    \begin{itemize}
        \item Assign higher weight to minority class in loss function
        \item XGBoost: \texttt{scale\_pos\_weight} parameter
    \end{itemize}

    \item \textbf{Evaluation Metrics:}
    \begin{itemize}
        \item Don't use accuracy (misleading)
        \item Use Precision@K, Recall@K, F1, AUC-PR
    \end{itemize}

    \item \textbf{Hard Negative Mining:}
    \begin{itemize}
        \item Sample negatives intelligently (products that were clicked but not ordered)
        \item Harder negatives improve model discrimination
    \end{itemize}
\end{enumerate}

\subsubsection{Question 3: NDCG vs. MAP}

\textit{When would you use NDCG instead of MAP for search evaluation?}

\textbf{Answer:}

\textbf{Use NDCG when:}
\begin{enumerate}
    \item \textbf{Graded relevance:} Products have different levels of relevance (highly relevant, somewhat relevant, not relevant)
    \item \textbf{Position matters a lot:} Top results should be weighted much more than lower results
    \item \textbf{Industry standard:} NDCG is more common in industry (Google, Bing, etc.)
\end{enumerate}

\textbf{Use MAP when:}
\begin{enumerate}
    \item \textbf{Binary relevance:} Items are either relevant or not (no grades)
    \item \textbf{All relevant items matter:} Not just top K
\end{enumerate}

\textbf{For Faire search:} Use NDCG@10 (graded relevance, top 10 results critical)

\subsection{ML System Design: Click Prediction Model}

\textit{Design an ML system to predict click probability for search results at Faire.}

\subsubsection{Problem Formulation}

\textbf{Input:} (retailer, query, product)

\textbf{Output:} P(click | retailer, query, product)

\textbf{Training Data:} Search logs (query, shown products, clicked products)

\subsubsection{Feature Engineering}

\textbf{Query Features:}
\begin{itemize}
    \item Query length, word count
    \item Query category (intent classification)
    \item Historical CTR for this query
\end{itemize}

\textbf{Product Features:}
\begin{itemize}
    \item Price, category, brand
    \item Popularity (orders, clicks, add-to-carts)
    \item Image quality score
    \item Inventory level
\end{itemize}

\textbf{Retailer Features:}
\begin{itemize}
    \item Store category, location
    \item Past order history (categories, price range)
    \item Account age, activity level
\end{itemize}

\textbf{Query-Product Features:}
\begin{itemize}
    \item Text match score (BM25)
    \item Embedding similarity (query embedding · product embedding)
    \item Position in search results (strong predictor!)
\end{itemize}

\textbf{Retailer-Product Features:}
\begin{itemize}
    \item Cosine similarity in embedding space
    \item Did retailer order this category before?
    \item Price relative to retailer's avg. order value
\end{itemize}

\subsubsection{Model Choice}

\textbf{Option 1: XGBoost (Faire's choice)}
\begin{itemize}
    \item \textbf{Pros:} Fast training, handles tabular features well, interpretable
    \item \textbf{Cons:} Doesn't capture complex interactions automatically
\end{itemize}

\textbf{Option 2: Deep Neural Network}
\begin{itemize}
    \item \textbf{Pros:} Learns feature interactions, works well with embeddings
    \item \textbf{Cons:} Slower, harder to interpret, needs more data
\end{itemize}

\textbf{Hybrid:} Use embeddings from DNN as features in XGBoost

\subsubsection{Training Pipeline}

\begin{enumerate}
    \item \textbf{Data Collection:} Stream search logs to data warehouse (Kinesis)
    \item \textbf{Feature Engineering:} Daily batch job (Spark) to compute features
    \item \textbf{Training:} Weekly re-train XGBoost on last 30 days of data
    \item \textbf{Evaluation:} Offline: AUC, LogLoss; Online: A/B test CTR
    \item \textbf{Deployment:} Export model to PMML, load in ranking service
\end{enumerate}

\subsubsection{Serving}

\begin{lstlisting}
def rank_search_results(retailer_id, query, candidates):
    # Get retailer features from cache
    retailer_features = redis.get(f"retailer:{retailer_id}")

    # Compute query features
    query_features = compute_query_features(query)

    # For each candidate product
    scored_products = []
    for product in candidates:
        # Get product features from cache
        product_features = redis.get(f"product:{product.id}")

        # Compute interaction features
        interaction_features = compute_interaction_features(
            retailer_features, query_features, product_features
        )

        # Combine all features
        feature_vector = concatenate([
            retailer_features,
            query_features,
            product_features,
            interaction_features
        ])

        # Predict click probability
        click_prob = xgboost_model.predict(feature_vector)

        scored_products.append((product, click_prob))

    # Sort by predicted click probability
    scored_products.sort(key=lambda x: x[1], reverse=True)

    return [p for p, score in scored_products]
\end{lstlisting}

\subsubsection{Position Bias Problem}

\textbf{Issue:} Items at top positions get more clicks simply due to position, not relevance.

\textbf{Solutions:}
\begin{enumerate}
    \item \textbf{Position feature:} Include position as feature (naive, doesn't fix bias)
    \item \textbf{Inverse Propensity Weighting:} Weight training examples by $1/P(\text{position})$
    \item \textbf{Randomization:} Randomly shuffle top results for small \% of traffic to collect unbiased data
    \item \textbf{Examination hypothesis:} Model P(click) = P(relevant) $\times$ P(examined)
\end{enumerate}

% ============================================================================
\section{Values Interview Prep (Interview Part 4)}
% ============================================================================

\textbf{Context:} This section covers the Values Interview (Part 4), a 30-minute behavioral interview focused on alignment with Faire's 5 core values.

\subsection{Faire's OFFICIAL Core Values}

These are the official values from the recruiter document. You MUST prepare stories for each value.

\subsubsection{1. Be an Owner}

\textbf{Sub-values:}
\begin{itemize}
    \item \textbf{Resourceful:} Find creative solutions with available resources
    \item \textbf{Raise the Bar:} Set and maintain high standards
    \item \textbf{Share Our Legos:} Build reusable systems, share knowledge
\end{itemize}

\textbf{Prepare a story about:}
\begin{itemize}
    \item Taking ownership of a complex problem end-to-end
    \item Being resourceful when faced with constraints
    \item Raising technical standards on your team
    \item Building reusable infrastructure or tools
\end{itemize}

\subsubsection{2. Be Kind}

\textbf{Sub-values:}
\begin{itemize}
    \item \textbf{Collaboration:} Work effectively with others
    \item \textbf{Coachability:} Accept feedback and grow
    \item \textbf{Inclusive:} Create welcoming environment for all
\end{itemize}

\textbf{Prepare a story about:}
\begin{itemize}
    \item A positive working relationship with a teammate
    \item A challenging working relationship you navigated
    \item Receiving feedback and how you improved
    \item Making your team more inclusive or collaborative
\end{itemize}

\subsubsection{3. Embrace the Adventure}

\textbf{Sub-values:}
\begin{itemize}
    \item \textbf{Agility:} Adapt quickly to change
    \item \textbf{Bias Towards Action:} Move fast, iterate, learn
\end{itemize}

\textbf{Prepare a story about:}
\begin{itemize}
    \item Thinking outside the box to solve a problem
    \item Adapting to a major change (requirements, tech stack, team)
    \item Taking action despite uncertainty
    \item Learning from failure and iterating quickly
\end{itemize}

\subsubsection{4. Seek the Truth}

\textbf{Sub-values:}
\begin{itemize}
    \item \textbf{Decision-Making:} Make data-driven decisions
    \item \textbf{Analytical Thinking:} Dig deep to understand root causes
    \item \textbf{Communication:} Communicate clearly and transparently
\end{itemize}

\textbf{Prepare a story about:}
\begin{itemize}
    \item Using data to make a critical decision
    \item Debugging a complex issue by digging into root causes
    \item Communicating transparently about a problem or failure
    \item Changing your mind based on evidence
\end{itemize}

\subsubsection{5. Serve Our Community}

\textbf{Sub-values:}
\begin{itemize}
    \item \textbf{Customer Centricity:} Put retailers and brands first
    \item \textbf{Prioritization:} Focus on highest-impact work
    \item \textbf{Motivation:} Inspired by Faire's mission
\end{itemize}

\textbf{Prepare a story about:}
\begin{itemize}
    \item Prioritizing customer needs over technical elegance
    \item Impact and influence - a high-impact project you led
    \item Why you care about Faire's mission (empowering entrepreneurs)
\end{itemize}

\subsection{Faire's Mission \& Scale (Know This Cold)}

\textbf{Mission:} Empower entrepreneurs everywhere to chase their dreams.

\textbf{Scale:}
\begin{itemize}
    \item 350,000+ independent retailers
    \item Independent retail sector larger than Amazon + Walmart combined
    \item 40\% of wholesale buying moved online during COVID
    \item Faire is uniquely positioned to serve this massive market
\end{itemize}

\textbf{Why it matters:} Small businesses are the backbone of local economies. Faire helps them compete.

\subsection{Common Behavioral Questions}

\subsubsection{Question 1: Why Faire?}

\textbf{Structure your answer:}
\begin{enumerate}
    \item \textbf{Mission alignment:} Why you care about supporting small businesses
    \item \textbf{Technical challenge:} Excited about search/ML/two-sided marketplace problems
    \item \textbf{Growth stage:} Opportunity to make impact at a scaling startup
    \item \textbf{Personal connection:} Maybe you have friends who run small shops
\end{enumerate}

\textbf{Example Answer:}

\textit{"I'm drawn to Faire because of the mission to empower independent retailers. My aunt runs a small boutique, and I've seen firsthand how challenging wholesale sourcing can be—long payment terms, high minimums, limited product discovery. Faire solves these pain points beautifully.}

\textit{From a technical perspective, I'm excited about the search and recommendation challenges. Building ML systems that balance retailer preferences with brand visibility is a fascinating two-sided optimization problem. And as someone passionate about data-driven product development, Faire's focus on experimentation and metrics really resonates with me."}

\subsubsection{Question 2: Tell me about a time you led a project}

\textbf{Use STAR method:}
\begin{itemize}
    \item \textbf{Situation:} Set the context
    \item \textbf{Task:} What needed to be done?
    \item \textbf{Action:} What did YOU do? (focus on your contributions)
    \item \textbf{Result:} Quantifiable impact
\end{itemize}

\textbf{Prepare examples for:}
\begin{itemize}
    \item Leading a technical project
    \item Solving a challenging bug
    \item Collaborating with non-engineers (PM, design)
    \item Making a data-driven decision
    \item Handling ambiguity or changing requirements
    \item Dealing with a difficult teammate or situation
\end{itemize}

\subsubsection{Question 3: What metrics would you track for Faire?}

This was mentioned as a common behavioral question. Show business acumen.

\textbf{Retailer Metrics:}
\begin{itemize}
    \item \textbf{Acquisition:} New retailer signups, activation rate
    \item \textbf{Engagement:} Monthly active retailers, searches per session
    \item \textbf{Conversion:} Order conversion rate, avg. order value
    \item \textbf{Retention:} Repeat purchase rate, churn rate, LTV
\end{itemize}

\textbf{Brand Metrics:}
\begin{itemize}
    \item \textbf{Acquisition:} New brand signups, catalog size
    \item \textbf{Engagement:} Order volume, revenue
    \item \textbf{Retention:} Brand churn, \% brands with repeat orders
\end{itemize}

\textbf{Platform Metrics:}
\begin{itemize}
    \item \textbf{GMV:} Gross Merchandise Value
    \item \textbf{Take rate:} Platform revenue as \% of GMV
    \item \textbf{Search quality:} Zero-result rate, CTR, conversion
\end{itemize}

\textbf{Two-Sided Health:}
\begin{itemize}
    \item \textbf{Liquidity:} \% retailers who find products to order
    \item \textbf{Balance:} New brands getting orders vs. established brands
\end{itemize}

\subsection{Questions to Ask Interviewers}

\textbf{About the Role:}
\begin{itemize}
    \item What does success look like in the first 3/6/12 months?
    \item What are the biggest technical challenges the team is facing?
    \item How does the team balance new features vs. tech debt?
\end{itemize}

\textbf{About the Team:}
\begin{itemize}
    \item How is the search team structured? (Frontend, backend, ML, data science?)
    \item How do you collaborate with PM, design, and data science?
    \item What's the on-call rotation like?
\end{itemize}

\textbf{About the Company:}
\begin{itemize}
    \item How has Faire's strategy evolved with the market?
    \item What are the company's top priorities for the next year?
    \item How do you balance retailer needs vs. brand needs in product decisions?
\end{itemize}

\textbf{Technical:}
\begin{itemize}
    \item What ML models are you currently using for search ranking?
    \item How do you evaluate new ranking algorithms? (A/B testing process?)
    \item What's the data infrastructure like? (Streaming vs. batch?)
    \item How do you handle cold start for new retailers/brands?
\end{itemize}

% ============================================================================
\section{Part 5: Preparation Strategy}
% ============================================================================

\subsection{4-Week Study Plan}

\subsubsection{Week 1: Algorithms \& Data Structures}

\textbf{Goal:} Solve 30 LeetCode problems (focus on medium difficulty)

\textbf{Daily:} 2 hours
\begin{itemize}
    \item Day 1-2: Arrays \& Hashing (10 problems)
    \item Day 3-4: Two Pointers, Sliding Window (10 problems)
    \item Day 5: Binary Search (5 problems)
    \item Day 6: Linked Lists (5 problems)
    \item Day 7: Review \& mock interview
\end{itemize}

\textbf{Resources:}
\begin{itemize}
    \item LeetCode Blind 75
    \item NeetCode (YouTube + roadmap)
    \item Tech Interview Handbook
\end{itemize}

\subsubsection{Week 2: More Algorithms + SQL}

\textbf{Goal:} 25 more problems + 10 SQL problems

\textbf{Daily:} 2 hours
\begin{itemize}
    \item Day 1-2: Trees (BFS, DFS, BST) - 10 problems
    \item Day 3: Graphs (BFS, DFS, Union-Find) - 5 problems
    \item Day 4: Dynamic Programming (5 problems)
    \item Day 5: SQL (10 problems on LeetCode)
    \item Day 6: Heaps, Stacks, Queues (5 problems)
    \item Day 7: Review \& mock interview
\end{itemize}

\subsubsection{Week 3: System Design}

\textbf{Goal:} Master 4 system design problems + read about Faire's stack

\textbf{Daily:} 2-3 hours
\begin{itemize}
    \item Day 1: Read Faire's engineering blog (craft.faire.com)
    \item Day 2: Design search system (practice this guide's Problem 1)
    \item Day 3: Design recommendation system (practice Problem 2)
    \item Day 4: Design URL shortener (classic problem)
    \item Day 5: Design Instagram/Twitter feed
    \item Day 6: Review all designs, focus on tradeoffs
    \item Day 7: Mock system design interview with friend
\end{itemize}

\textbf{Resources:}
\begin{itemize}
    \item Grokking the System Design Interview
    \item System Design Primer (GitHub)
    \item ByteByteGo (YouTube)
\end{itemize}

\subsubsection{Week 4: ML + Behavioral + Mock Interviews}

\textbf{Goal:} Solidify ML metrics, prepare stories, simulate interviews

\textbf{Daily:} 2-3 hours
\begin{itemize}
    \item Day 1: Study ML metrics (this guide's Part 3)
    \item Day 2: Practice ML system design (click prediction, ranking)
    \item Day 3: Prepare behavioral stories (STAR method, 5 stories)
    \item Day 4: Mock coding interview (Pramp or friend)
    \item Day 5: Mock system design interview
    \item Day 6: Review all weak areas
    \item Day 7: Rest, light review, prepare questions to ask
\end{itemize}

\subsection{Day Before Interview}

\begin{itemize}
    \item \textbf{Light review:} Skim your notes, don't cram
    \item \textbf{Re-read this guide:} Sections 1, 2, 3 summaries
    \item \textbf{Prepare questions:} Write down 5 questions to ask
    \item \textbf{Logistics:} Test video/audio, charge laptop, have pen \& paper ready
    \item \textbf{Mindset:} Get good sleep, stay confident!
\end{itemize}

\subsection{During the Interview}

\textbf{Coding Interview:}
\begin{enumerate}
    \item \textbf{Clarify:} Ask questions, confirm input/output, edge cases
    \item \textbf{Think aloud:} Explain your approach before coding
    \item \textbf{Start simple:} Brute force first, then optimize
    \item \textbf{Test:} Walk through examples, catch bugs
    \item \textbf{Communicate:} If stuck, ask for hints (shows collaboration)
\end{enumerate}

\textbf{System Design Interview:}
\begin{enumerate}
    \item \textbf{Clarify requirements:} Functional, non-functional, scale
    \item \textbf{High-level design first:} Draw boxes (API, DB, cache, etc.)
    \item \textbf{Deep dive:} Pick 2-3 components to detail (interviewer will guide)
    \item \textbf{Discuss tradeoffs:} SQL vs. NoSQL, sync vs. async, etc.
    \item \textbf{Time management:} Don't spend 40 min on API design
\end{enumerate}

\textbf{Behavioral Interview:}
\begin{enumerate}
    \item \textbf{STAR method:} Structure answers clearly
    \item \textbf{Quantify impact:} "Reduced latency by 40\%", "Increased CTR by 15\%"
    \item \textbf{Show learning:} "This taught me to...", "I'd do X differently next time"
    \item \textbf{Ask questions:} Show genuine interest in the role
\end{enumerate}

% ============================================================================
\section{Appendix A: Faire-Specific Questions (To Be Added)}
% ============================================================================

\textit{This section will be populated with actual interview questions from Glassdoor, Prepfully, and InterviewQuery after you provide the content.}

\subsection{Placeholder for Glassdoor Questions}

\textit{[To be filled in after user provides Glassdoor content]}

\subsection{Placeholder for Prepfully Questions}

\textit{[To be filled in after user provides Prepfully content]}

\subsection{Placeholder for InterviewQuery Content}

\textit{[To be filled in after user provides InterviewQuery content]}

% ============================================================================
\section{Appendix B: Faire Engineering Blog Insights}
% ============================================================================

\subsection{Key Articles to Read}

\begin{enumerate}
    \item \textbf{"How data and machine learning shape Faire's marketplace"}
    \begin{itemize}
        \item URL: craft.faire.com/how-data-and-machine-learning-shape-faires-marketplace-510855c4a9bc
        \item \textbf{Key takeaways:}
        \begin{itemize}
            \item Recommendation embeddings from clicks, add-to-cart, orders
            \item Hybrid recommender system for cold start
            \item XGBoost models for ranking
            \item Redis feature store for real-time features
            \item Balancing retailer and brand stakeholder needs
        \end{itemize}
    \end{itemize}

    \item \textbf{"Real-time ranking at Faire part 2: the feature store"}
    \begin{itemize}
        \item URL: craft.faire.com/real-time-ranking-at-faire-part-2-the-feature-store-3f1013d3fe5d
        \item \textbf{Key takeaways:}
        \begin{itemize}
            \item Feature store architecture with Redis
            \item Handling hundreds of features per product
            \item Low-latency feature retrieval (< 10ms)
        \end{itemize}
    \end{itemize}
\end{enumerate}

\subsection{Tech Stack Summary}

\textbf{Languages:} Kotlin (backend), JavaScript/TypeScript (frontend)

\textbf{Data:} MySQL, Redis, NoSQL, ElasticSearch, Kinesis, DBT

\textbf{ML:} XGBoost, embeddings, collaborative filtering

\textbf{Infrastructure:} Kubernetes, AWS (ELB, SQS), Nginx

% ============================================================================
\section{Appendix C: Additional Resources}
% ============================================================================

\subsection{Coding Practice}

\begin{itemize}
    \item \textbf{LeetCode:} leetcode.com (Blind 75, NeetCode 150)
    \item \textbf{NeetCode:} neetcode.io (video solutions + roadmap)
    \item \textbf{AlgoExpert:} algoexpert.io (paid, high quality)
    \item \textbf{HackerRank:} hackerrank.com (SQL practice)
\end{itemize}

\subsection{System Design}

\begin{itemize}
    \item \textbf{Grokking the System Design Interview:} educative.io
    \item \textbf{System Design Primer:} github.com/donnemartin/system-design-primer
    \item \textbf{ByteByteGo:} bytebytego.com (Alex Xu's course)
    \item \textbf{Designing Data-Intensive Applications:} Book by Martin Kleppmann
\end{itemize}

\subsection{Machine Learning}

\begin{itemize}
    \item \textbf{StatQuest:} youtube.com/@statquest (ML concepts)
    \item \textbf{Google ML Crash Course:} developers.google.com/machine-learning/crash-course
    \item \textbf{Chip Huyen's ML Interviews Book:} huyenchip.com/ml-interviews-book
\end{itemize}

\subsection{Behavioral}

\begin{itemize}
    \item \textbf{Tech Interview Handbook:} techinterviewhandbook.org/behavioral-interview
    \item \textbf{Grokking the Behavioral Interview:} educative.io
\end{itemize}

% ============================================================================
\section{Summary \& Final Tips}
% ============================================================================

\subsection{Key Takeaways}

\begin{enumerate}
    \item \textbf{Faire is a two-sided marketplace} connecting retailers and brands via search and recommendations

    \item \textbf{Tech stack:} Kotlin, ElasticSearch, XGBoost, Redis, MySQL, Kubernetes

    \item \textbf{Interview focus:}
    \begin{itemize}
        \item Algorithms: Medium LeetCode, 2 rounds DSA
        \item System Design: Search systems (your main focus)
        \item ML: Evaluation metrics (precision, recall, NDCG)
        \item Behavioral: Culture fit, why Faire
    \end{itemize}

    \item \textbf{Search system design} is critical - know two-stage retrieval + ranking

    \item \textbf{ML metrics:} Master NDCG, Precision@K, Recall@K for ranking problems

    \item \textbf{Tradeoffs:} Be ready to discuss latency vs. accuracy, precision vs. recall, complexity vs. maintainability
\end{enumerate}

\subsection{The Night Before}

\begin{itemize}
    \item Review this guide's summaries (sections 2, 4, 5)
    \item Re-read Faire's mission and engineering blog posts
    \item Prepare 5 questions to ask
    \item Get 8 hours of sleep
\end{itemize}

\subsection{You've Got This!}

You've prepared thoroughly. Remember:
\begin{itemize}
    \item \textbf{Communication > perfection:} Interviewers want to see your thought process
    \item \textbf{Ask questions:} Shows curiosity and collaboration
    \item \textbf{Stay calm:} If stuck, take a breath, think aloud, ask for hints
    \item \textbf{Be yourself:} Faire values authenticity and passion
\end{itemize}

\textbf{Good luck! You're going to do great! 🚀}

\end{document}
