\documentclass[10pt,a4paper]{article}
\usepackage[utf8]{inputenc}
\usepackage[english]{babel}
\usepackage{amsmath}
\usepackage{amsfonts}
\usepackage{amssymb}
\usepackage{geometry}
\usepackage{fancyhdr}
\usepackage{listings}
\usepackage{xcolor}
\usepackage{tcolorbox}
\usepackage{enumitem}
\usepackage{hyperref}
\usepackage{titlesec}
\usepackage{array}
\usepackage{longtable}
\usepackage{booktabs}

% Page setup
\geometry{left=2cm,right=2cm,top=2.5cm,bottom=2.5cm}
\pagestyle{fancy}
\fancyhf{}
\fancyhead[L]{Coding Interview Patterns}
\fancyhead[R]{\thepage}
\fancyfoot[C]{Practice Until Perfect - Master These Templates}

% Python syntax highlighting
\definecolor{codegreen}{rgb}{0,0.6,0}
\definecolor{codegray}{rgb}{0.5,0.5,0.5}
\definecolor{codepurple}{rgb}{0.58,0,0.82}
\definecolor{backcolour}{rgb}{0.95,0.95,0.92}

\lstdefinestyle{pythonstyle}{
    backgroundcolor=\color{backcolour},
    commentstyle=\color{codegreen},
    keywordstyle=\color{magenta},
    numberstyle=\tiny\color{codegray},
    stringstyle=\color{codepurple},
    basicstyle=\ttfamily\footnotesize,
    breakatwhitespace=false,
    breaklines=true,
    captionpos=b,
    keepspaces=true,
    numbers=left,
    numbersep=5pt,
    showspaces=false,
    showstringspaces=false,
    showtabs=false,
    tabsize=2,
    language=Python
}

\lstset{style=pythonstyle}

% Custom boxes
\newtcolorbox{patternbox}[1]{
    colback=blue!5!white,
    colframe=blue!75!black,
    fonttitle=\bfseries,
    title=#1
}

\newtcolorbox{tipbox}{
    colback=green!5!white,
    colframe=green!75!black,
    fonttitle=\bfseries,
    title=Pro Tip
}

\newtcolorbox{complexitybox}{
    colback=orange!5!white,
    colframe=orange!75!black,
    fonttitle=\bfseries,
    title=Complexity Analysis
}

% Title formatting
\titleformat{\section}{\Large\bfseries\color{blue!80!black}}{\thesection}{1em}{}
\titleformat{\subsection}{\large\bfseries\color{blue!60!black}}{\thesubsection}{1em}{}

\begin{document}

% Title Page
\begin{titlepage}
\centering
{\huge\bfseries Coding Interview Patterns \& Templates\par}
\vspace{1cm}
{\LARGE Python Reference Guide\par}
\vspace{1.5cm}
{\Large Master These Patterns for Interview Success\par}
\vspace{2cm}

\begin{tcolorbox}[colback=blue!10,colframe=blue!50!black,width=0.8\textwidth]
\centering
\textbf{GOAL}: Practice these templates until they become muscle memory\\
\textbf{COVERAGE}: 12 Essential Patterns + Recognition Guide\\
\textbf{TIMELINE}: 4-Week Intensive Study Plan\\
\textbf{TARGET}: FAANG/MAANG Interview Readiness
\end{tcolorbox}

\vspace{2cm}
{\large Based on 2024 FAANG Interview Analysis\par}
\vfill
{\large \today\par}
\end{titlepage}

% Table of Contents
\tableofcontents
\newpage

% Quick Reference Card
\section{Quick Pattern Recognition Guide}

\begin{table}[h]
\centering
\begin{tabular}{|p{3cm}|p{4cm}|p{7cm}|}
\hline
\textbf{Pattern} & \textbf{Key Indicators} & \textbf{When to Use} \\
\hline
Two Pointers & Sorted array, pairs, palindromes & Finding pairs with target sum, removing duplicates \\
\hline
Sliding Window & Subarray/substring, contiguous & Max/min subarray, longest substring problems \\
\hline
DFS & Tree/graph traversal, explore all paths & Path finding, connectivity, topological sort \\
\hline
BFS & Shortest path, level-order & Minimum steps, level traversal \\
\hline
Binary Search & Sorted data, search space & Finding element, first/last occurrence \\
\hline
Dynamic Programming & Optimal substructure, overlapping subproblems & Optimization, counting, decision problems \\
\hline
Backtracking & Generate all combinations/permutations & N-Queens, Sudoku, subset generation \\
\hline
Heap & Top K, merge K, priority & Finding extremes, scheduling \\
\hline
Union-Find & Connected components, cycles & Graph connectivity, MST \\
\hline
Trie & Prefix matching, word games & Autocomplete, word search \\
\hline
Intervals & Overlapping ranges, scheduling & Meeting rooms, merge intervals \\
\hline
Matrix Traversal & 2D grid problems, connected components & Islands, flood fill \\
\hline
Tree Construction & Build tree from traversals & Preorder/inorder, serialize \\
\hline
\end{tabular}
\end{table}

\section{Core Algorithm Patterns}

\subsection{1. Depth-First Search (DFS)}

\begin{patternbox}{DFS - Tree Traversal}
\textbf{Use Case:} Tree/graph traversal, path finding, cycle detection\\
\textbf{Time:} O(V + E) for graphs, O(n) for trees\\
\textbf{Space:} O(h) where h is height/depth
\end{patternbox}

\subsubsection{Recursive DFS Template}
\begin{lstlisting}
def dfs_recursive(root):
    if not root:
        return  # Base case

    # Process current node
    print(root.val)

    # Recurse on children
    dfs_recursive(root.left)
    dfs_recursive(root.right)

# With result collection
def dfs_collect_paths(root, path=[], all_paths=[]):
    if not root:
        return

    path.append(root.val)

    # Leaf node - save path
    if not root.left and not root.right:
        all_paths.append(path[:])  # Copy path

    dfs_collect_paths(root.left, path, all_paths)
    dfs_collect_paths(root.right, path, all_paths)

    path.pop()  # Backtrack
    return all_paths
\end{lstlisting}

\subsubsection{Iterative DFS Template}
\begin{lstlisting}
def dfs_iterative(root):
    if not root:
        return

    stack = [root]
    while stack:
        node = stack.pop()
        print(node.val)  # Process node

        # Add children (right first for left-to-right order)
        if node.right:
            stack.append(node.right)
        if node.left:
            stack.append(node.left)

# Graph DFS with visited set
def dfs_graph(graph, start):
    visited = set()
    stack = [start]

    while stack:
        node = stack.pop()
        if node not in visited:
            visited.add(node)
            print(node)  # Process node

            # Add neighbors
            for neighbor in graph[node]:
                if neighbor not in visited:
                    stack.append(neighbor)
\end{lstlisting}

\subsection{2. Breadth-First Search (BFS)}

\begin{patternbox}{BFS - Level-by-Level Traversal}
\textbf{Use Case:} Shortest path, level-order traversal, minimum steps\\
\textbf{Time:} O(V + E) for graphs, O(n) for trees\\
\textbf{Space:} O(w) where w is maximum width
\end{patternbox}

\subsubsection{Standard BFS Template}
\begin{lstlisting}
from collections import deque

def bfs_level_order(root):
    if not root:
        return []

    result = []
    queue = deque([root])

    while queue:
        level_size = len(queue)
        level = []

        for _ in range(level_size):
            node = queue.popleft()
            level.append(node.val)

            if node.left:
                queue.append(node.left)
            if node.right:
                queue.append(node.right)

        result.append(level)
    return result

# Simple BFS without levels
def bfs_simple(root):
    if not root:
        return

    queue = deque([root])
    while queue:
        node = queue.popleft()
        print(node.val)  # Process node

        if node.left:
            queue.append(node.left)
        if node.right:
            queue.append(node.right)
\end{lstlisting}

\subsubsection{BFS for Shortest Path}
\begin{lstlisting}
def bfs_shortest_path(graph, start, target):
    queue = deque([(start, 0)])  # (node, distance)
    visited = set([start])

    while queue:
        node, dist = queue.popleft()

        if node == target:
            return dist

        for neighbor in graph[node]:
            if neighbor not in visited:
                visited.add(neighbor)
                queue.append((neighbor, dist + 1))

    return -1  # Target not reachable
\end{lstlisting}

\subsection{3. Two Pointers}

\begin{patternbox}{Two Pointers - Efficient Array/String Processing}
\textbf{Use Case:} Sorted arrays, palindromes, pairs with target sum\\
\textbf{Time:} O(n) typically\\
\textbf{Space:} O(1) space optimization
\end{patternbox}

\subsubsection{Opposite Ends Template}
\begin{lstlisting}
# Problem: Two Sum II - Input Array Is Sorted
# Given a sorted array, find two numbers that add up to target
# Return indices of the two numbers (assume exactly one solution exists)
def two_sum_sorted(nums, target):
    left, right = 0, len(nums) - 1

    while left < right:
        current_sum = nums[left] + nums[right]

        if current_sum == target:
            return [left, right]
        elif current_sum < target:
            left += 1
        else:
            right -= 1

    return []

# Problem: Valid Palindrome
# Given a string, determine if it's a palindrome (reads same forwards/backwards)
# Consider only alphanumeric characters and ignore case
def is_palindrome(s):
    left, right = 0, len(s) - 1

    while left < right:
        if s[left] != s[right]:
            return False
        left += 1
        right -= 1

    return True

# Problem: Reverse Array
# Reverse an array in-place using two pointers
def reverse_array(arr):
    left, right = 0, len(arr) - 1

    while left < right:
        arr[left], arr[right] = arr[right], arr[left]
        left += 1
        right -= 1
\end{lstlisting}

\subsubsection{Fast and Slow Pointers}
\begin{lstlisting}
# Problem: Linked List Cycle (Floyd's Cycle Detection)
# Detect if a linked list has a cycle using fast/slow pointers
def has_cycle(head):
    if not head or not head.next:
        return False

    slow = fast = head

    while fast and fast.next:
        slow = slow.next
        fast = fast.next.next

        if slow == fast:
            return True

    return False

# Problem: Middle of the Linked List
# Find the middle node of a linked list
def find_middle(head):
    slow = fast = head

    while fast and fast.next:
        slow = slow.next
        fast = fast.next.next

    return slow

# Problem: Remove Nth Node From End of List
# Remove the nth node from the end of a linked list
def remove_nth_from_end(head, n):
    dummy = ListNode(0)
    dummy.next = head
    slow = fast = dummy

    # Move fast n+1 steps ahead
    for _ in range(n + 1):
        fast = fast.next

    # Move both until fast reaches end
    while fast:
        slow = slow.next
        fast = fast.next

    # Remove nth node
    slow.next = slow.next.next
    return dummy.next
\end{lstlisting}

\subsection{4. Sliding Window}

\begin{patternbox}{Sliding Window - Contiguous Subarray/Substring}
\textbf{Use Case:} Maximum/minimum subarray, longest substring\\
\textbf{Time:} O(n) single pass\\
\textbf{Space:} O(k) for tracking window contents
\end{patternbox}

\subsubsection{Fixed Size Window}
\begin{lstlisting}
# Problem: Maximum Sum Subarray of Size K
# Find the maximum sum of any contiguous subarray of size k
def max_sum_subarray(nums, k):
    if len(nums) < k:
        return 0

    # Calculate first window
    window_sum = sum(nums[:k])
    max_sum = window_sum

    # Slide window
    for i in range(k, len(nums)):
        window_sum = window_sum - nums[i - k] + nums[i]
        max_sum = max(max_sum, window_sum)

    return max_sum

def find_averages(nums, k):
    result = []
    window_sum = 0

    for i in range(len(nums)):
        window_sum += nums[i]

        if i >= k - 1:
            result.append(window_sum / k)
            window_sum -= nums[i - k + 1]

    return result
\end{lstlisting}

\subsubsection{Variable Size Window}
\begin{lstlisting}
# Problem: Longest Substring with At Most K Distinct Characters
# Find the length of the longest substring with at most k distinct characters
def longest_substring_k_distinct(s, k):
    if k == 0:
        return 0

    char_count = {}
    left = 0
    max_length = 0

    for right in range(len(s)):
        # Expand window
        char_count[s[right]] = char_count.get(s[right], 0) + 1

        # Contract window if needed
        while len(char_count) > k:
            char_count[s[left]] -= 1
            if char_count[s[left]] == 0:
                del char_count[s[left]]
            left += 1

        max_length = max(max_length, right - left + 1)

    return max_length

def min_window_substring(s, t):
    if not s or not t:
        return ""

    t_count = {}
    for char in t:
        t_count[char] = t_count.get(char, 0) + 1

    left = 0
    min_len = float('inf')
    min_start = 0
    matched = 0
    window_count = {}

    for right in range(len(s)):
        char = s[right]
        window_count[char] = window_count.get(char, 0) + 1

        if char in t_count and window_count[char] == t_count[char]:
            matched += 1

        while matched == len(t_count):
            if right - left + 1 < min_len:
                min_len = right - left + 1
                min_start = left

            left_char = s[left]
            window_count[left_char] -= 1
            if left_char in t_count and window_count[left_char] < t_count[left_char]:
                matched -= 1
            left += 1

    return s[min_start:min_start + min_len] if min_len != float('inf') else ""
\end{lstlisting}

\subsection{5. Binary Search}

\begin{patternbox}{Binary Search - Efficient Search in Sorted Space}
\textbf{Use Case:} Finding element, first/last occurrence, search space\\
\textbf{Time:} O(log n)\\
\textbf{Space:} O(1) iterative, O(log n) recursive
\end{patternbox}

\subsubsection{Standard Binary Search}
\begin{lstlisting}
# Problem: Binary Search
# Find target value in a sorted array, return index or -1 if not found
def binary_search(nums, target):
    left, right = 0, len(nums) - 1

    while left <= right:
        mid = left + (right - left) // 2

        if nums[mid] == target:
            return mid
        elif nums[mid] < target:
            left = mid + 1
        else:
            right = mid - 1

    return -1

def binary_search_recursive(nums, target, left=0, right=None):
    if right is None:
        right = len(nums) - 1

    if left > right:
        return -1

    mid = left + (right - left) // 2

    if nums[mid] == target:
        return mid
    elif nums[mid] < target:
        return binary_search_recursive(nums, target, mid + 1, right)
    else:
        return binary_search_recursive(nums, target, left, mid - 1)
\end{lstlisting}

\subsubsection{Find First and Last Occurrence}
\begin{lstlisting}
def find_first_occurrence(nums, target):
    left, right = 0, len(nums) - 1
    result = -1

    while left <= right:
        mid = left + (right - left) // 2

        if nums[mid] == target:
            result = mid
            right = mid - 1  # Continue searching left
        elif nums[mid] < target:
            left = mid + 1
        else:
            right = mid - 1

    return result

def find_last_occurrence(nums, target):
    left, right = 0, len(nums) - 1
    result = -1

    while left <= right:
        mid = left + (right - left) // 2

        if nums[mid] == target:
            result = mid
            left = mid + 1  # Continue searching right
        elif nums[mid] < target:
            left = mid + 1
        else:
            right = mid - 1

    return result

def search_range(nums, target):
    return [find_first_occurrence(nums, target),
            find_last_occurrence(nums, target)]
\end{lstlisting}

\subsubsection{Binary Search on Answer}
\begin{lstlisting}
def find_peak_element(nums):
    left, right = 0, len(nums) - 1

    while left < right:
        mid = left + (right - left) // 2

        if nums[mid] > nums[mid + 1]:
            right = mid
        else:
            left = mid + 1

    return left

def search_rotated_sorted_array(nums, target):
    left, right = 0, len(nums) - 1

    while left <= right:
        mid = left + (right - left) // 2

        if nums[mid] == target:
            return mid

        # Left half is sorted
        if nums[left] <= nums[mid]:
            if nums[left] <= target < nums[mid]:
                right = mid - 1
            else:
                left = mid + 1
        # Right half is sorted
        else:
            if nums[mid] < target <= nums[right]:
                left = mid + 1
            else:
                right = mid - 1

    return -1
\end{lstlisting}

\subsection{6. Dynamic Programming}

\begin{patternbox}{Dynamic Programming - Optimal Substructure}
\textbf{Use Case:} Optimization, counting, decision problems\\
\textbf{Time:} Varies (often O(n²) or O(n*m))\\
\textbf{Space:} O(n) to O(n*m) depending on dimensions
\end{patternbox}

\subsubsection{1D DP Template}
\begin{lstlisting}
# Problem: Fibonacci Number
# Calculate the nth Fibonacci number using dynamic programming
def fibonacci(n):
    if n <= 1:
        return n

    # Bottom-up approach
    dp = [0] * (n + 1)
    dp[1] = 1

    for i in range(2, n + 1):
        dp[i] = dp[i-1] + dp[i-2]

    return dp[n]

# Space optimized
def fibonacci_optimized(n):
    if n <= 1:
        return n

    prev2, prev1 = 0, 1

    for i in range(2, n + 1):
        current = prev1 + prev2
        prev2, prev1 = prev1, current

    return prev1

def climb_stairs(n):
    if n <= 2:
        return n

    dp = [0] * (n + 1)
    dp[1], dp[2] = 1, 2

    for i in range(3, n + 1):
        dp[i] = dp[i-1] + dp[i-2]

    return dp[n]

def house_robber(nums):
    if not nums:
        return 0
    if len(nums) == 1:
        return nums[0]

    dp = [0] * len(nums)
    dp[0] = nums[0]
    dp[1] = max(nums[0], nums[1])

    for i in range(2, len(nums)):
        dp[i] = max(dp[i-1], dp[i-2] + nums[i])

    return dp[-1]
\end{lstlisting}

\subsubsection{2D DP Template}
\begin{lstlisting}
def unique_paths(m, n):
    # Create DP table
    dp = [[1 for _ in range(n)] for _ in range(m)]

    for i in range(1, m):
        for j in range(1, n):
            dp[i][j] = dp[i-1][j] + dp[i][j-1]

    return dp[m-1][n-1]

def longest_common_subsequence(text1, text2):
    m, n = len(text1), len(text2)
    dp = [[0 for _ in range(n + 1)] for _ in range(m + 1)]

    for i in range(1, m + 1):
        for j in range(1, n + 1):
            if text1[i-1] == text2[j-1]:
                dp[i][j] = dp[i-1][j-1] + 1
            else:
                dp[i][j] = max(dp[i-1][j], dp[i][j-1])

    return dp[m][n]

def min_path_sum(grid):
    m, n = len(grid), len(grid[0])

    # Initialize first row and column
    for i in range(1, m):
        grid[i][0] += grid[i-1][0]

    for j in range(1, n):
        grid[0][j] += grid[0][j-1]

    # Fill the DP table
    for i in range(1, m):
        for j in range(1, n):
            grid[i][j] += min(grid[i-1][j], grid[i][j-1])

    return grid[m-1][n-1]
\end{lstlisting}

\subsection{7. Backtracking}

\begin{patternbox}{Backtracking - Explore All Possibilities}
\textbf{Use Case:} Permutations, combinations, N-Queens, Sudoku\\
\textbf{Time:} Exponential O($2^n$) or O($n!$)\\
\textbf{Space:} O(depth) for recursion stack
\end{patternbox}

\subsubsection{General Backtracking Template}
\begin{lstlisting}
def backtrack_template(candidates, target, path=[], result=[]):
    # Base case - solution found
    if is_valid_solution(path, target):
        result.append(path[:])  # Make a copy
        return

    # Explore all possibilities
    for i, candidate in enumerate(candidates):
        # Skip invalid candidates
        if not is_valid_candidate(candidate, path):
            continue

        # Make choice
        path.append(candidate)

        # Recurse with remaining candidates
        backtrack_template(candidates[i+1:], target, path, result)

        # Undo choice (backtrack)
        path.pop()

    return result

def generate_permutations(nums):
    def backtrack(path, remaining):
        if not remaining:
            result.append(path[:])
            return

        for i in range(len(remaining)):
            # Choose
            path.append(remaining[i])
            # Explore
            backtrack(path, remaining[:i] + remaining[i+1:])
            # Unchoose
            path.pop()

    result = []
    backtrack([], nums)
    return result
\end{lstlisting}

\subsubsection{Combinations and Subsets}
\begin{lstlisting}
def generate_subsets(nums):
    def backtrack(start, path):
        result.append(path[:])  # Add current subset

        for i in range(start, len(nums)):
            path.append(nums[i])
            backtrack(i + 1, path)
            path.pop()

    result = []
    backtrack(0, [])
    return result

def combination_sum(candidates, target):
    def backtrack(start, path, remaining):
        if remaining == 0:
            result.append(path[:])
            return

        for i in range(start, len(candidates)):
            if candidates[i] > remaining:
                break

            path.append(candidates[i])
            # Can reuse same element
            backtrack(i, path, remaining - candidates[i])
            path.pop()

    result = []
    candidates.sort()
    backtrack(0, [], target)
    return result

def letter_combinations(digits):
    if not digits:
        return []

    phone = {
        '2': 'abc', '3': 'def', '4': 'ghi', '5': 'jkl',
        '6': 'mno', '7': 'pqrs', '8': 'tuv', '9': 'wxyz'
    }

    def backtrack(index, path):
        if index == len(digits):
            result.append("".join(path))
            return

        for letter in phone[digits[index]]:
            path.append(letter)
            backtrack(index + 1, path)
            path.pop()

    result = []
    backtrack(0, [])
    return result
\end{lstlisting}

\subsection{8. Heap Operations}

\begin{patternbox}{Heap - Priority Queue for Top K Problems}
\textbf{Use Case:} Top K elements, merge K sorted, scheduling\\
\textbf{Time:} O(log n) insert/delete, O(1) peek\\
\textbf{Space:} O(n) for heap storage
\end{patternbox}

\subsubsection{Min/Max Heap Templates}
\begin{lstlisting}
import heapq

def find_kth_largest(nums, k):
    # Use min heap of size k
    min_heap = []

    for num in nums:
        heapq.heappush(min_heap, num)
        if len(min_heap) > k:
            heapq.heappop(min_heap)

    return min_heap[0]

def find_k_largest_elements(nums, k):
    # Min heap approach
    min_heap = []

    for num in nums:
        if len(min_heap) < k:
            heapq.heappush(min_heap, num)
        elif num > min_heap[0]:
            heapq.heapreplace(min_heap, num)

    return sorted(min_heap, reverse=True)

# For max heap, negate values
def max_heap_operations():
    max_heap = []

    # Insert (negate for max heap)
    heapq.heappush(max_heap, -value)

    # Extract max (negate result)
    max_val = -heapq.heappop(max_heap)

    # Peek max
    max_val = -max_heap[0]
\end{lstlisting}

\subsubsection{Advanced Heap Applications}
\begin{lstlisting}
def merge_k_sorted_lists(lists):
    import heapq

    min_heap = []
    # Add first element from each list
    for i, lst in enumerate(lists):
        if lst:
            heapq.heappush(min_heap, (lst.val, i, lst))

    dummy = ListNode(0)
    current = dummy

    while min_heap:
        val, list_idx, node = heapq.heappop(min_heap)
        current.next = node
        current = current.next

        # Add next element from same list
        if node.next:
            heapq.heappush(min_heap, (node.next.val, list_idx, node.next))

    return dummy.next

def top_k_frequent_elements(nums, k):
    from collections import Counter
    import heapq

    count = Counter(nums)

    # Min heap of size k
    min_heap = []
    for num, freq in count.items():
        heapq.heappush(min_heap, (freq, num))
        if len(min_heap) > k:
            heapq.heappop(min_heap)

    return [num for freq, num in min_heap]

class MedianFinder:
    def __init__(self):
        self.small = []  # max heap (negated)
        self.large = []  # min heap

    def addNum(self, num):
        # Add to max heap first
        heapq.heappush(self.small, -num)

        # Balance: move largest from small to large
        heapq.heappush(self.large, -heapq.heappop(self.small))

        # Ensure small has more or equal elements
        if len(self.large) > len(self.small):
            heapq.heappush(self.small, -heapq.heappop(self.large))

    def findMedian(self):
        if len(self.small) > len(self.large):
            return -self.small[0]
        return (-self.small[0] + self.large[0]) / 2
\end{lstlisting}

\subsection{9. Union-Find (Disjoint Set)}

\begin{patternbox}{Union-Find - Connected Components}
\textbf{Use Case:} Graph connectivity, cycle detection, MST\\
\textbf{Time:} O($\alpha(n)$) amortized per operation\\
\textbf{Space:} O(n) for parent and rank arrays
\end{patternbox}

\begin{lstlisting}
class UnionFind:
    def __init__(self, n):
        self.parent = list(range(n))
        self.rank = [0] * n
        self.components = n

    def find(self, x):
        # Path compression
        if self.parent[x] != x:
            self.parent[x] = self.find(self.parent[x])
        return self.parent[x]

    def union(self, x, y):
        root_x = self.find(x)
        root_y = self.find(y)

        if root_x == root_y:
            return False  # Already connected

        # Union by rank
        if self.rank[root_x] < self.rank[root_y]:
            self.parent[root_x] = root_y
        elif self.rank[root_x] > self.rank[root_y]:
            self.parent[root_y] = root_x
        else:
            self.parent[root_y] = root_x
            self.rank[root_x] += 1

        self.components -= 1
        return True

    def connected(self, x, y):
        return self.find(x) == self.find(y)

    def count_components(self):
        return self.components

# Applications
def number_of_islands(grid):
    if not grid:
        return 0

    m, n = len(grid), len(grid[0])
    uf = UnionFind(m * n)
    islands = 0

    for i in range(m):
        for j in range(n):
            if grid[i][j] == '1':
                islands += 1
                # Check 4 directions
                for di, dj in [(0,1), (1,0), (0,-1), (-1,0)]:
                    ni, nj = i + di, j + dj
                    if (0 <= ni < m and 0 <= nj < n and
                        grid[ni][nj] == '1'):
                        if uf.union(i*n + j, ni*n + nj):
                            islands -= 1

    return islands

def redundant_connection(edges):
    uf = UnionFind(len(edges) + 1)

    for u, v in edges:
        if not uf.union(u, v):
            return [u, v]  # This edge creates a cycle

    return []
\end{lstlisting}

\subsection{10. Trie (Prefix Tree)}

\begin{patternbox}{Trie - Efficient String Storage and Search}
\textbf{Use Case:} Autocomplete, word search, prefix matching\\
\textbf{Time:} O(m) where m is string length\\
\textbf{Space:} O(ALPHABET\_SIZE * N * M) worst case
\end{patternbox}

\begin{lstlisting}
class TrieNode:
    def __init__(self):
        self.children = {}
        self.is_end_of_word = False

class Trie:
    def __init__(self):
        self.root = TrieNode()

    def insert(self, word):
        node = self.root
        for char in word:
            if char not in node.children:
                node.children[char] = TrieNode()
            node = node.children[char]
        node.is_end_of_word = True

    def search(self, word):
        node = self.root
        for char in word:
            if char not in node.children:
                return False
            node = node.children[char]
        return node.is_end_of_word

    def starts_with(self, prefix):
        node = self.root
        for char in prefix:
            if char not in node.children:
                return False
            node = node.children[char]
        return True

    def find_words_with_prefix(self, prefix):
        # Find the prefix node
        node = self.root
        for char in prefix:
            if char not in node.children:
                return []
            node = node.children[char]

        # DFS to find all words
        words = []
        self._dfs(node, prefix, words)
        return words

    def _dfs(self, node, path, words):
        if node.is_end_of_word:
            words.append(path)

        for char, child_node in node.children.items():
            self._dfs(child_node, path + char, words)

# Word Search II using Trie
def find_words_in_board(board, words):
    # Build trie
    trie = Trie()
    for word in words:
        trie.insert(word)

    result = set()
    m, n = len(board), len(board[0])

    def dfs(i, j, node, path):
        if node.is_end_of_word:
            result.add(path)

        if (i < 0 or i >= m or j < 0 or j >= n or
            board[i][j] not in node.children):
            return

        char = board[i][j]
        board[i][j] = '#'  # Mark visited

        # Explore 4 directions
        for di, dj in [(0,1), (1,0), (0,-1), (-1,0)]:
            dfs(i + di, j + dj, node.children[char], path + char)

        board[i][j] = char  # Restore

    # Try starting from each cell
    for i in range(m):
        for j in range(n):
            if board[i][j] in trie.root.children:
                dfs(i, j, trie.root, "")

    return list(result)
\end{lstlisting}

\subsection{11. Topological Sort}

\begin{patternbox}{Topological Sort - Ordering with Dependencies}
\textbf{Use Case:} Course scheduling, dependency resolution\\
\textbf{Time:} O(V + E)\\
\textbf{Space:} O(V) for in-degree array and queue
\end{patternbox}

\begin{lstlisting}
from collections import defaultdict, deque

def topological_sort_kahn(num_courses, prerequisites):
    # Build graph and in-degree array
    graph = defaultdict(list)
    in_degree = [0] * num_courses

    for course, prereq in prerequisites:
        graph[prereq].append(course)
        in_degree[course] += 1

    # Initialize queue with nodes having in-degree 0
    queue = deque([i for i in range(num_courses) if in_degree[i] == 0])
    result = []

    while queue:
        node = queue.popleft()
        result.append(node)

        # Reduce in-degree of neighbors
        for neighbor in graph[node]:
            in_degree[neighbor] -= 1
            if in_degree[neighbor] == 0:
                queue.append(neighbor)

    # Check if topological sort is possible
    return result if len(result) == num_courses else []

def can_finish_courses(num_courses, prerequisites):
    topo_order = topological_sort_kahn(num_courses, prerequisites)
    return len(topo_order) == num_courses

# DFS-based topological sort
def topological_sort_dfs(graph):
    visited = set()
    rec_stack = set()
    result = []

    def dfs(node):
        if node in rec_stack:
            return False  # Cycle detected
        if node in visited:
            return True

        visited.add(node)
        rec_stack.add(node)

        for neighbor in graph.get(node, []):
            if not dfs(neighbor):
                return False

        rec_stack.remove(node)
        result.append(node)  # Add to result in post-order
        return True

    for node in graph:
        if node not in visited:
            if not dfs(node):
                return []  # Cycle detected

    return result[::-1]  # Reverse for correct order
\end{lstlisting}

\subsection{12. Interval Problems}

\begin{patternbox}{Interval Problems - Merge and Insert Intervals}
\textbf{Use Case:} Meeting rooms, overlapping intervals, scheduling\\
\textbf{Time:} O(n log n) for sorting, O(n) for merging\\
\textbf{Space:} O(n) for result storage
\end{patternbox}

\begin{lstlisting}
# Problem: Merge Intervals
# Given collection of intervals, merge overlapping intervals
def merge_intervals(intervals):
    if not intervals:
        return []

    # Sort by start time
    intervals.sort(key=lambda x: x[0])
    merged = [intervals[0]]

    for current in intervals[1:]:
        last = merged[-1]

        # Overlapping intervals
        if current[0] <= last[1]:
            # Merge by updating end time
            last[1] = max(last[1], current[1])
        else:
            # Non-overlapping, add to result
            merged.append(current)

    return merged

# Problem: Insert Interval
# Insert new interval and merge if necessary
def insert_interval(intervals, new_interval):
    result = []
    i = 0
    n = len(intervals)

    # Add all intervals before new_interval
    while i < n and intervals[i][1] < new_interval[0]:
        result.append(intervals[i])
        i += 1

    # Merge overlapping intervals
    while i < n and intervals[i][0] <= new_interval[1]:
        new_interval[0] = min(new_interval[0], intervals[i][0])
        new_interval[1] = max(new_interval[1], intervals[i][1])
        i += 1

    result.append(new_interval)

    # Add remaining intervals
    while i < n:
        result.append(intervals[i])
        i += 1

    return result

# Problem: Meeting Rooms II
# Find minimum number of meeting rooms needed
def min_meeting_rooms(intervals):
    if not intervals:
        return 0

    # Separate start and end times
    starts = sorted([interval[0] for interval in intervals])
    ends = sorted([interval[1] for interval in intervals])

    rooms = 0
    end_ptr = 0

    for start in starts:
        # If meeting starts after another ends, reuse room
        if start >= ends[end_ptr]:
            end_ptr += 1
        else:
            # Need new room
            rooms += 1

    return rooms
\end{lstlisting}

\subsection{13. Matrix Traversal}

\begin{patternbox}{Matrix Problems - 2D Grid Traversal}
\textbf{Use Case:} Island problems, path finding, flood fill\\
\textbf{Time:} O(m*n) where m,n are matrix dimensions\\
\textbf{Space:} O(m*n) for visited array or recursion stack
\end{patternbox}

\begin{lstlisting}
# Problem: Number of Islands
# Count connected components of '1's in 2D grid
def num_islands(grid):
    if not grid or not grid[0]:
        return 0

    m, n = len(grid), len(grid[0])
    count = 0

    def dfs(i, j):
        # Boundary check and water check
        if (i < 0 or i >= m or j < 0 or j >= n or
            grid[i][j] != '1'):
            return

        # Mark as visited
        grid[i][j] = '0'

        # Explore 4 directions
        for di, dj in [(0,1), (1,0), (0,-1), (-1,0)]:
            dfs(i + di, j + dj)

    for i in range(m):
        for j in range(n):
            if grid[i][j] == '1':
                dfs(i, j)
                count += 1

    return count

# Problem: Rotting Oranges (Multi-source BFS)
# Find time for all oranges to rot
def oranges_rotting(grid):
    from collections import deque

    m, n = len(grid), len(grid[0])
    queue = deque()
    fresh = 0

    # Find all rotten oranges and count fresh ones
    for i in range(m):
        for j in range(n):
            if grid[i][j] == 2:
                queue.append((i, j))
            elif grid[i][j] == 1:
                fresh += 1

    if fresh == 0:
        return 0

    time = 0
    directions = [(0,1), (1,0), (0,-1), (-1,0)]

    while queue:
        time += 1
        size = len(queue)

        for _ in range(size):
            x, y = queue.popleft()

            for dx, dy in directions:
                nx, ny = x + dx, y + dy

                if (0 <= nx < m and 0 <= ny < n and
                    grid[nx][ny] == 1):
                    grid[nx][ny] = 2
                    fresh -= 1
                    queue.append((nx, ny))

    return time - 1 if fresh == 0 else -1
\end{lstlisting}

\subsection{14. Tree Construction}

\begin{patternbox}{Tree Construction - Build Trees from Traversals}
\textbf{Use Case:} Construct tree from preorder/inorder, serialize/deserialize\\
\textbf{Time:} O(n) with hashmap optimization\\
\textbf{Space:} O(n) for hashmap and recursion stack
\end{patternbox}

\begin{lstlisting}
# Problem: Construct Binary Tree from Preorder and Inorder
# Build tree given preorder and inorder traversal arrays
def build_tree_preorder_inorder(preorder, inorder):
    if not preorder or not inorder:
        return None

    # Build hashmap for O(1) inorder lookups
    inorder_map = {val: i for i, val in enumerate(inorder)}
    self.preorder_idx = 0

    def build(left, right):
        if left > right:
            return None

        # Root is current preorder element
        root_val = preorder[self.preorder_idx]
        self.preorder_idx += 1
        root = TreeNode(root_val)

        # Find root position in inorder
        root_idx = inorder_map[root_val]

        # Build left subtree first (preorder property)
        root.left = build(left, root_idx - 1)
        root.right = build(root_idx + 1, right)

        return root

    return build(0, len(inorder) - 1)

# Problem: Serialize and Deserialize Binary Tree
# Convert tree to string and back
def serialize(root):
    def preorder(node):
        if not node:
            result.append("null")
            return

        result.append(str(node.val))
        preorder(node.left)
        preorder(node.right)

    result = []
    preorder(root)
    return ",".join(result)

def deserialize(data):
    def build():
        val = next(values)
        if val == "null":
            return None

        node = TreeNode(int(val))
        node.left = build()
        node.right = build()
        return node

    values = iter(data.split(","))
    return build()
\end{lstlisting}

\subsection{15. Special Algorithms}

\subsubsection{Kadane's Algorithm - Maximum Subarray}
\begin{lstlisting}
def max_subarray_sum(nums):
    if not nums:
        return 0

    max_sum = current_sum = nums[0]

    for i in range(1, len(nums)):
        # Either extend existing subarray or start new one
        current_sum = max(nums[i], current_sum + nums[i])
        max_sum = max(max_sum, current_sum)

    return max_sum

def max_subarray_with_indices(nums):
    max_sum = current_sum = nums[0]
    start = end = temp_start = 0

    for i in range(1, len(nums)):
        if current_sum < 0:
            current_sum = nums[i]
            temp_start = i
        else:
            current_sum += nums[i]

        if current_sum > max_sum:
            max_sum = current_sum
            start = temp_start
            end = i

    return max_sum, start, end
\end{lstlisting}

\subsubsection{Prefix Sum}
\begin{lstlisting}
class PrefixSum:
    def __init__(self, nums):
        self.prefix = [0]
        for num in nums:
            self.prefix.append(self.prefix[-1] + num)

    def range_sum(self, i, j):
        # Sum from index i to j (inclusive)
        return self.prefix[j + 1] - self.prefix[i]

def subarray_sum_equals_k(nums, k):
    count = 0
    prefix_sum = 0
    sum_count = {0: 1}  # prefix_sum -> frequency

    for num in nums:
        prefix_sum += num
        if prefix_sum - k in sum_count:
            count += sum_count[prefix_sum - k]
        sum_count[prefix_sum] = sum_count.get(prefix_sum, 0) + 1

    return count
\end{lstlisting}

\subsubsection{Monotonic Stack}
\begin{lstlisting}
def next_greater_element(nums):
    result = [-1] * len(nums)
    stack = []  # Store indices

    for i in range(len(nums)):
        while stack and nums[i] > nums[stack[-1]]:
            index = stack.pop()
            result[index] = nums[i]
        stack.append(i)

    return result

def daily_temperatures(temperatures):
    result = [0] * len(temperatures)
    stack = []

    for i, temp in enumerate(temperatures):
        while stack and temp > temperatures[stack[-1]]:
            prev_index = stack.pop()
            result[prev_index] = i - prev_index
        stack.append(i)

    return result

def largest_rectangle_in_histogram(heights):
    stack = []
    max_area = 0

    for i, h in enumerate(heights + [0]):  # Add sentinel
        while stack and h < heights[stack[-1]]:
            height = heights[stack.pop()]
            width = i if not stack else i - stack[-1] - 1
            max_area = max(max_area, height * width)
        stack.append(i)

    return max_area
\end{lstlisting}

\subsubsection{Cyclic Sort}
\begin{lstlisting}
def cyclic_sort(nums):
    i = 0
    while i < len(nums):
        correct_index = nums[i] - 1
        if nums[i] != nums[correct_index]:
            nums[i], nums[correct_index] = nums[correct_index], nums[i]
        else:
            i += 1
    return nums

def find_missing_number(nums):
    # Array contains n numbers in range [0, n]
    i = 0
    n = len(nums)

    while i < n:
        if nums[i] < n and nums[i] != nums[nums[i]]:
            nums[nums[i]], nums[i] = nums[i], nums[nums[i]]
        else:
            i += 1

    # Find the missing number
    for i in range(n):
        if nums[i] != i:
            return i

    return n

def find_all_duplicates(nums):
    duplicates = []

    for i in range(len(nums)):
        # Use array indices to mark presence
        num = abs(nums[i])
        if nums[num - 1] < 0:
            duplicates.append(num)
        else:
            nums[num - 1] *= -1

    return duplicates
\end{lstlisting}

\section{Pattern Recognition Guide}

\begin{complexitybox}
\textbf{Time Complexity Quick Reference:}
\begin{itemize}
\item O(1) - Hash table access, array index
\item O(log n) - Binary search, heap operations
\item O(n) - Single pass through array, BFS/DFS
\item O(n log n) - Merge sort, heap sort
\item O(n²) - Nested loops, bubble sort
\item O($2^n$) - Recursive algorithms without memoization
\end{itemize}
\end{complexitybox}

\subsection{Decision Tree for Pattern Selection}

\begin{enumerate}
\item \textbf{Array/String Problems:}
   \begin{itemize}
   \item Sorted array $\rightarrow$ Binary Search
   \item Two elements with condition $\rightarrow$ Two Pointers
   \item Contiguous subarray/substring $\rightarrow$ Sliding Window
   \item All subarrays $\rightarrow$ Prefix Sum or DP
   \end{itemize}

\item \textbf{Tree/Graph Problems:}
   \begin{itemize}
   \item Path finding $\rightarrow$ DFS
   \item Shortest path $\rightarrow$ BFS
   \item Level-order traversal $\rightarrow$ BFS
   \item Connected components $\rightarrow$ Union-Find or DFS
   \end{itemize}

\item \textbf{Optimization Problems:}
   \begin{itemize}
   \item Optimal substructure $\rightarrow$ Dynamic Programming
   \item Multiple choices at each step $\rightarrow$ Backtracking
   \item Greedy choice property $\rightarrow$ Greedy Algorithm
   \end{itemize}

\item \textbf{Priority/Ordering Problems:}
   \begin{itemize}
   \item Top K elements $\rightarrow$ Heap
   \item Streaming data $\rightarrow$ Heap
   \item Dependencies $\rightarrow$ Topological Sort
   \end{itemize}

\item \textbf{String Matching:}
   \begin{itemize}
   \item Prefix operations $\rightarrow$ Trie
   \item Pattern matching $\rightarrow$ KMP or Rolling Hash
   \item Anagrams $\rightarrow$ Hash Map
   \end{itemize}
\end{enumerate}

\begin{tipbox}
\textbf{Red Flags for Common Patterns:}
\begin{itemize}
\item "Maximum/Minimum subarray" $\rightarrow$ Sliding Window or Kadane's
\item "All permutations/combinations" $\rightarrow$ Backtracking
\item "Shortest path in unweighted graph" $\rightarrow$ BFS
\item "Detect cycle" $\rightarrow$ DFS or Union-Find
\item "Top K" or "K-th largest" $\rightarrow$ Heap
\end{itemize}
\end{tipbox}

\section{4-Week Study Schedule}

\subsection{Week 1: Foundation Patterns}
\textbf{Day 1-2:} Two Pointers \& Sliding Window
\begin{itemize}
\item Practice writing templates from memory
\item Solve: Two Sum II, Container With Most Water, Longest Substring Without Repeating Characters
\end{itemize}

\textbf{Day 3-4:} DFS \& BFS
\begin{itemize}
\item Master both recursive and iterative approaches
\item Solve: Binary Tree Inorder Traversal, Binary Tree Level Order Traversal, Number of Islands
\end{itemize}

\textbf{Day 5-7:} Binary Search
\begin{itemize}
\item Practice standard and modified binary search
\item Solve: Search in Rotated Sorted Array, Find First and Last Position, Search Insert Position
\end{itemize}

\subsection{Week 2: Dynamic Programming \& Backtracking}
\textbf{Day 8-10:} 1D \& 2D Dynamic Programming
\begin{itemize}
\item Focus on identifying optimal substructure
\item Solve: Climbing Stairs, House Robber, Unique Paths, Longest Common Subsequence
\end{itemize}

\textbf{Day 11-14:} Backtracking
\begin{itemize}
\item Master the template and understand when to backtrack
\item Solve: Permutations, Combinations, Subsets, N-Queens
\end{itemize}

\subsection{Week 3: Advanced Data Structures}
\textbf{Day 15-17:} Heap Operations
\begin{itemize}
\item Learn min/max heap operations and applications
\item Solve: Kth Largest Element, Top K Frequent Elements, Merge K Sorted Lists
\end{itemize}

\textbf{Day 18-19:} Union-Find
\begin{itemize}
\item Understand path compression and union by rank
\item Solve: Number of Islands II, Redundant Connection, Friend Circles
\end{itemize}

\textbf{Day 20-21:} Trie
\begin{itemize}
\item Build trie from scratch and understand applications
\item Solve: Implement Trie, Word Search II, Add and Search Word
\end{itemize}

\subsection{Week 4: Special Algorithms \& Integration}
\textbf{Day 22-24:} Special Algorithms
\begin{itemize}
\item Kadane's Algorithm, Prefix Sum, Monotonic Stack
\item Solve: Maximum Subarray, Subarray Sum Equals K, Next Greater Element
\end{itemize}

\textbf{Day 25-26:} Topological Sort
\begin{itemize}
\item Both Kahn's algorithm and DFS approach
\item Solve: Course Schedule, Course Schedule II, Alien Dictionary
\end{itemize}

\textbf{Day 27-28:} Integration \& Mock Interviews
\begin{itemize}
\item Combine multiple patterns in complex problems
\item Practice writing templates under time pressure
\item Mock interview sessions
\end{itemize}

\section{Daily Practice Routine}

\subsection{Morning Routine (30 minutes)}
\begin{enumerate}
\item \textbf{Template Writing (10 min):} Write 3 random templates from memory
\item \textbf{Pattern Recognition (10 min):} Look at problem titles and identify patterns
\item \textbf{Complexity Analysis (10 min):} Review time/space complexity for each pattern
\end{enumerate}

\subsection{Evening Session (60-90 minutes)}
\begin{enumerate}
\item \textbf{Problem Solving (45-60 min):}
   \begin{itemize}
   \item Choose 2-3 problems focusing on current week's patterns
   \item Spend 5 minutes identifying the pattern before coding
   \item Write solution from scratch using templates
   \end{itemize}
\item \textbf{Template Review (15-30 min):}
   \begin{itemize}
   \item Review any templates you struggled with
   \item Write them again from memory
   \item Note common mistakes or variations
   \end{itemize}
\end{enumerate}

\subsection{Weekly Goals}
\begin{itemize}
\item \textbf{Week 1:} Master 4 foundational patterns
\item \textbf{Week 2:} Add DP and backtracking to arsenal
\item \textbf{Week 3:} Comfortable with advanced data structures
\item \textbf{Week 4:} Integrate patterns and achieve fluency
\end{itemize}

\begin{tipbox}
\textbf{Success Metrics:}
\begin{itemize}
\item Can write any template from memory in under 2 minutes
\item Identify correct pattern within 30 seconds of reading problem
\item Solve medium problems in 15-20 minutes
\item Explain time/space complexity confidently
\end{itemize}
\end{tipbox}

\section{Progress Tracking}

\subsection{Template Mastery Checklist}
Check off when you can write from memory in under 2 minutes:

\begin{itemize}
\item[$\square$] DFS (recursive)
\item[$\square$] DFS (iterative)
\item[$\square$] BFS (standard)
\item[$\square$] BFS (level-order)
\item[$\square$] Two Pointers (opposite ends)
\item[$\square$] Two Pointers (fast/slow)
\item[$\square$] Sliding Window (fixed)
\item[$\square$] Sliding Window (variable)
\item[$\square$] Binary Search (standard)
\item[$\square$] Binary Search (first/last occurrence)
\item[$\square$] 1D DP template
\item[$\square$] 2D DP template
\item[$\square$] Backtracking template
\item[$\square$] Heap operations
\item[$\square$] Union-Find (with optimizations)
\item[$\square$] Trie implementation
\item[$\square$] Topological Sort (Kahn's)
\item[$\square$] Kadane's Algorithm
\item[$\square$] Prefix Sum
\item[$\square$] Monotonic Stack
\end{itemize}

\subsection{Problem Categories to Master}
\begin{table}[h]
\centering
\begin{tabular}{|l|c|c|c|}
\hline
\textbf{Category} & \textbf{Easy} & \textbf{Medium} & \textbf{Hard} \\
\hline
Arrays \& Strings & 10 & 15 & 5 \\
\hline
Trees \& Graphs & 8 & 12 & 5 \\
\hline
Dynamic Programming & 5 & 10 & 5 \\
\hline
Backtracking & 3 & 8 & 3 \\
\hline
Heaps \& Priority Queues & 5 & 8 & 3 \\
\hline
Advanced Data Structures & 5 & 10 & 5 \\
\hline
\textbf{Total} & \textbf{36} & \textbf{63} & \textbf{26} \\
\hline
\end{tabular}
\end{table}

\section{Final Tips for Success}

\begin{tipbox}
\textbf{Interview Day Strategy:}
\begin{enumerate}
\item Spend 2-3 minutes understanding the problem completely
\item Identify the pattern (this should be automatic by now)
\item Explain your approach before coding
\item Write the template structure first, then fill in details
\item Test with simple examples
\item Analyze time and space complexity
\end{enumerate}
\end{tipbox}

\begin{complexitybox}
\textbf{Common Optimizations:}
\begin{itemize}
\item Two-pass $\rightarrow$ One-pass with hash map
\item Nested loops $\rightarrow$ Two pointers or hash map
\item Recursion $\rightarrow$ DP with memoization
\item Multiple data structures $\rightarrow$ Single optimized structure
\item Extra space $\rightarrow$ In-place modifications
\end{itemize}
\end{complexitybox}

\subsection{Red Flags to Avoid}
\begin{itemize}
\item Jumping into coding without understanding the problem
\item Not identifying the pattern before starting
\item Overcomplicating simple problems
\item Not testing with edge cases
\item Forgetting to analyze complexity
\end{itemize}

\subsection{Last-Minute Review (Day Before Interview)}
\begin{enumerate}
\item Write all 12 main templates from memory
\item Review the pattern recognition guide
\item Practice one problem from each category
\item Get good sleep and stay confident
\end{enumerate}

\vspace{1cm}
\begin{center}
\textbf{\Large Remember: Consistency beats intensity.}\\
\textbf{\Large Practice these templates daily until they become automatic!}
\end{center}

\newpage
\appendix
\section{Daily Progress Tracking Worksheets}

\subsection{Template Mastery Progress Tracker}

\begin{center}
\textbf{\Large Week \_\_\_\_ Progress Tracker}\\
\textbf{Date Range: \_\_\_\_\_\_\_\_\_\_ to \_\_\_\_\_\_\_\_\_\_}
\end{center}

\subsubsection{Daily Template Practice (Write from Memory)}

\begin{table}[h]
\centering
\begin{tabular}{|p{4cm}|c|c|c|c|c|c|c|}
\hline
\textbf{Template} & \textbf{Mon} & \textbf{Tue} & \textbf{Wed} & \textbf{Thu} & \textbf{Fri} & \textbf{Sat} & \textbf{Sun} \\
\hline
DFS Recursive & $\square$ & $\square$ & $\square$ & $\square$ & $\square$ & $\square$ & $\square$ \\
\hline
DFS Iterative & $\square$ & $\square$ & $\square$ & $\square$ & $\square$ & $\square$ & $\square$ \\
\hline
BFS Standard & $\square$ & $\square$ & $\square$ & $\square$ & $\square$ & $\square$ & $\square$ \\
\hline
Two Pointers (Opposite) & $\square$ & $\square$ & $\square$ & $\square$ & $\square$ & $\square$ & $\square$ \\
\hline
Two Pointers (Fast/Slow) & $\square$ & $\square$ & $\square$ & $\square$ & $\square$ & $\square$ & $\square$ \\
\hline
Sliding Window (Fixed) & $\square$ & $\square$ & $\square$ & $\square$ & $\square$ & $\square$ & $\square$ \\
\hline
Sliding Window (Variable) & $\square$ & $\square$ & $\square$ & $\square$ & $\square$ & $\square$ & $\square$ \\
\hline
Binary Search & $\square$ & $\square$ & $\square$ & $\square$ & $\square$ & $\square$ & $\square$ \\
\hline
Binary Search (First/Last) & $\square$ & $\square$ & $\square$ & $\square$ & $\square$ & $\square$ & $\square$ \\
\hline
1D DP Template & $\square$ & $\square$ & $\square$ & $\square$ & $\square$ & $\square$ & $\square$ \\
\hline
2D DP Template & $\square$ & $\square$ & $\square$ & $\square$ & $\square$ & $\square$ & $\square$ \\
\hline
Backtracking Template & $\square$ & $\square$ & $\square$ & $\square$ & $\square$ & $\square$ & $\square$ \\
\hline
\end{tabular}
\end{table}

\textbf{Goal:} Check $\square$ if you can write the template from memory in under 2 minutes

\subsubsection{Problem Solving Progress}

\begin{table}[h]
\centering
\begin{tabular}{|p{3cm}|p{5cm}|p{2cm}|p{2cm}|p{2cm}|}
\hline
\textbf{Date} & \textbf{Problem Name} & \textbf{Pattern} & \textbf{Difficulty} & \textbf{Time (min)} \\
\hline
\_\_\_\_\_\_\_\_ & \_\_\_\_\_\_\_\_\_\_\_\_\_\_\_\_\_\_\_\_\_\_\_\_\_\_\_\_\_\_\_\_ & \_\_\_\_\_\_\_\_\_\_ & Easy/Med/Hard & \_\_\_\_\_ \\
\hline
\_\_\_\_\_\_\_\_ & \_\_\_\_\_\_\_\_\_\_\_\_\_\_\_\_\_\_\_\_\_\_\_\_\_\_\_\_\_\_\_\_ & \_\_\_\_\_\_\_\_\_\_ & Easy/Med/Hard & \_\_\_\_\_ \\
\hline
\_\_\_\_\_\_\_\_ & \_\_\_\_\_\_\_\_\_\_\_\_\_\_\_\_\_\_\_\_\_\_\_\_\_\_\_\_\_\_\_\_ & \_\_\_\_\_\_\_\_\_\_ & Easy/Med/Hard & \_\_\_\_\_ \\
\hline
\_\_\_\_\_\_\_\_ & \_\_\_\_\_\_\_\_\_\_\_\_\_\_\_\_\_\_\_\_\_\_\_\_\_\_\_\_\_\_\_\_ & \_\_\_\_\_\_\_\_\_\_ & Easy/Med/Hard & \_\_\_\_\_ \\
\hline
\_\_\_\_\_\_\_\_ & \_\_\_\_\_\_\_\_\_\_\_\_\_\_\_\_\_\_\_\_\_\_\_\_\_\_\_\_\_\_\_\_ & \_\_\_\_\_\_\_\_\_\_ & Easy/Med/Hard & \_\_\_\_\_ \\
\hline
\_\_\_\_\_\_\_\_ & \_\_\_\_\_\_\_\_\_\_\_\_\_\_\_\_\_\_\_\_\_\_\_\_\_\_\_\_\_\_\_\_ & \_\_\_\_\_\_\_\_\_\_ & Easy/Med/Hard & \_\_\_\_\_ \\
\hline
\_\_\_\_\_\_\_\_ & \_\_\_\_\_\_\_\_\_\_\_\_\_\_\_\_\_\_\_\_\_\_\_\_\_\_\_\_\_\_\_\_ & \_\_\_\_\_\_\_\_\_\_ & Easy/Med/Hard & \_\_\_\_\_ \\
\hline
\_\_\_\_\_\_\_\_ & \_\_\_\_\_\_\_\_\_\_\_\_\_\_\_\_\_\_\_\_\_\_\_\_\_\_\_\_\_\_\_\_ & \_\_\_\_\_\_\_\_\_\_ & Easy/Med/Hard & \_\_\_\_\_ \\
\hline
\_\_\_\_\_\_\_\_ & \_\_\_\_\_\_\_\_\_\_\_\_\_\_\_\_\_\_\_\_\_\_\_\_\_\_\_\_\_\_\_\_ & \_\_\_\_\_\_\_\_\_\_ & Easy/Med/Hard & \_\_\_\_\_ \\
\hline
\_\_\_\_\_\_\_\_ & \_\_\_\_\_\_\_\_\_\_\_\_\_\_\_\_\_\_\_\_\_\_\_\_\_\_\_\_\_\_\_\_ & \_\_\_\_\_\_\_\_\_\_ & Easy/Med/Hard & \_\_\_\_\_ \\
\hline
\end{tabular}
\end{table}

\newpage

\subsection{4-Week Study Schedule Tracker}

\subsubsection{Week 1: Foundation Patterns}
\textbf{Focus:} Two Pointers, Sliding Window, DFS, BFS, Binary Search

\begin{table}[h]
\centering
\begin{tabular}{|p{2cm}|p{4cm}|p{6cm}|c|}
\hline
\textbf{Day} & \textbf{Primary Focus} & \textbf{Goals} & \textbf{Done} \\
\hline
Day 1-2 & Two Pointers \& Sliding Window & Master templates, solve: Two Sum II, Container With Most Water & $\square$ \\
\hline
Day 3-4 & DFS \& BFS & Recursive/iterative approaches, solve: Binary Tree Traversals & $\square$ \\
\hline
Day 5-7 & Binary Search & Standard and modified, solve: Search in Rotated Array & $\square$ \\
\hline
\end{tabular}
\end{table}

\textbf{Week 1 Notes:}\\
\fbox{\begin{minipage}{0.9\textwidth}
\vspace{1.5cm}
\end{minipage}}

\subsubsection{Week 2: Dynamic Programming \& Backtracking}
\textbf{Focus:} 1D \& 2D DP, Backtracking patterns

\begin{table}[h]
\centering
\begin{tabular}{|p{2cm}|p{4cm}|p{6cm}|c|}
\hline
\textbf{Day} & \textbf{Primary Focus} & \textbf{Goals} & \textbf{Done} \\
\hline
Day 8-10 & Dynamic Programming & 1D/2D templates, solve: Climbing Stairs, Unique Paths & $\square$ \\
\hline
Day 11-14 & Backtracking & Template mastery, solve: Permutations, N-Queens & $\square$ \\
\hline
\end{tabular}
\end{table}

\textbf{Week 2 Notes:}\\
\fbox{\begin{minipage}{0.9\textwidth}
\vspace{1.5cm}
\end{minipage}}

\subsubsection{Week 3: Advanced Data Structures}
\textbf{Focus:} Heaps, Union-Find, Trie, Intervals, Matrix

\begin{table}[h]
\centering
\begin{tabular}{|p{2cm}|p{4cm}|p{6cm}|c|}
\hline
\textbf{Day} & \textbf{Primary Focus} & \textbf{Goals} & \textbf{Done} \\
\hline
Day 15-17 & Heap Operations & Min/max heap applications, solve: Top K Elements & $\square$ \\
\hline
Day 18-19 & Union-Find & Path compression, solve: Number of Islands & $\square$ \\
\hline
Day 20-21 & Trie \& Intervals & Build from scratch, solve: Merge Intervals & $\square$ \\
\hline
\end{tabular}
\end{table}

\textbf{Week 3 Notes:}\\
\fbox{\begin{minipage}{0.9\textwidth}
\vspace{1.5cm}
\end{minipage}}

\subsubsection{Week 4: Special Algorithms \& Integration}
\textbf{Focus:} Special algorithms, pattern combinations, mock interviews

\begin{table}[h]
\centering
\begin{tabular}{|p{2cm}|p{4cm}|p{6cm}|c|}
\hline
\textbf{Day} & \textbf{Primary Focus} & \textbf{Goals} & \textbf{Done} \\
\hline
Day 22-24 & Special Algorithms & Kadane's, Prefix Sum, Monotonic Stack & $\square$ \\
\hline
Day 25-26 & Topological Sort & Kahn's algorithm, solve: Course Schedule & $\square$ \\
\hline
Day 27-28 & Integration \& Mocks & Combine patterns, timed practice sessions & $\square$ \\
\hline
\end{tabular}
\end{table}

\textbf{Week 4 Notes:}\\
\fbox{\begin{minipage}{0.9\textwidth}
\vspace{1.5cm}
\end{minipage}}

\newpage

\subsection{Scientific Spaced Repetition Schedule}

\textbf{Optimal Learning and Review Cadence for Maximum Retention}

This schedule follows cognitive science principles with review intervals at 1 day, 3 days, 7 days, and 14 days to maximize long-term retention.

\subsubsection{Week 1: Foundation Building}

\begin{table}[h]
\centering
\small
\begin{tabular}{|c|p{3cm}|p{4cm}|p{4cm}|c|}
\hline
\textbf{Day} & \textbf{New Learning} & \textbf{Learning Activity} & \textbf{Review (Spaced Intervals)} & \textbf{Done} \\
\hline
Day 1 & Two Pointers & Learn template, solve: Two Sum II & -- & $\square$ \\
\hline
Day 2 & Sliding Window & Learn template, solve: Longest Substring & Two Pointers (1-day) & $\square$ \\
\hline
Day 3 & Binary Search & Learn template, solve: Search in Rotated Array & Sliding Window (1-day) & $\square$ \\
\hline
Day 4 & DFS Fundamentals & Learn tree/graph traversal & Two Pointers (3-day), Binary Search (1-day) & $\square$ \\
\hline
Day 5 & BFS Fundamentals & Learn level-order traversal & Sliding Window (3-day), DFS (1-day) & $\square$ \\
\hline
Day 6 & Dynamic Programming & Learn basic DP patterns & Binary Search (3-day), BFS (1-day) & $\square$ \\
\hline
Day 7 & Week 1 Integration & Combine learned patterns & DFS (3-day), DP (1-day) & $\square$ \\
\hline
\end{tabular}
\end{table}

\subsubsection{Week 2: Pattern Reinforcement}

\begin{table}[h]
\centering
\small
\begin{tabular}{|c|p{3cm}|p{4cm}|p{4cm}|c|}
\hline
\textbf{Day} & \textbf{New Learning} & \textbf{Learning Activity} & \textbf{Review (Spaced Intervals)} & \textbf{Done} \\
\hline
Day 8 & Advanced DFS & Backtracking, path finding & Two Pointers (7-day), BFS (3-day), DP (1-day) & $\square$ \\
\hline
Day 9 & Advanced DP & 2D DP, optimization & Sliding Window (7-day), Advanced DFS (1-day) & $\square$ \\
\hline
Day 10 & Greedy Algorithms & Local optimal choices & Binary Search (7-day), Advanced DP (1-day) & $\square$ \\
\hline
Day 11 & Backtracking & Permutations, combinations & DFS (7-day), Greedy (1-day) & $\square$ \\
\hline
Day 12 & Advanced Backtracking & N-Queens, Sudoku patterns & BFS (7-day), Backtracking (1-day) & $\square$ \\
\hline
Day 13 & Interval Problems & Merge, overlap detection & DP (7-day), Advanced Backtracking (1-day) & $\square$ \\
\hline
Day 14 & Week 2 Integration & Complex problem combinations & Week 1 Integration (7-day), Intervals (1-day) & $\square$ \\
\hline
\end{tabular}
\end{table}

\subsubsection{Week 3: Advanced Structures}

\begin{table}[h]
\centering
\small
\begin{tabular}{|c|p{3cm}|p{4cm}|p{4cm}|c|}
\hline
\textbf{Day} & \textbf{New Learning} & \textbf{Learning Activity} & \textbf{Review (Spaced Intervals)} & \textbf{Done} \\
\hline
Day 15 & Heap Operations & Min/max heaps, K problems & Advanced DFS (7-day), Week 2 Integration (1-day) & $\square$ \\
\hline
Day 16 & Priority Queues & Advanced heap applications & Advanced DP (7-day), Heaps (1-day) & $\square$ \\
\hline
Day 17 & Union-Find & Path compression, optimization & Greedy (7-day), Priority Queues (1-day) & $\square$ \\
\hline
Day 18 & Trie Operations & Prefix trees, word problems & Backtracking (7-day), Union-Find (1-day) & $\square$ \\
\hline
Day 19 & Advanced Trie & Auto-complete, word search & Advanced Backtracking (7-day), Trie (1-day) & $\square$ \\
\hline
Day 20 & Matrix Traversal & 2D patterns, connectivity & Intervals (7-day), Advanced Trie (1-day) & $\square$ \\
\hline
Day 21 & Week 3 Integration & Advanced data structure combos & Two Pointers (14-day), Matrix (1-day) & $\square$ \\
\hline
\end{tabular}
\end{table}

\subsubsection{Week 4: Mastery and Integration}

\begin{table}[h]
\centering
\small
\begin{tabular}{|c|p{3cm}|p{4cm}|p{4cm}|c|}
\hline
\textbf{Day} & \textbf{New Learning} & \textbf{Learning Activity} & \textbf{Review (Spaced Intervals)} & \textbf{Done} \\
\hline
Day 22 & Special Algorithms & Kadane's, KMP, Manacher's & Sliding Window (14-day), Heaps (7-day), Week 3 Integration (1-day) & $\square$ \\
\hline
Day 23 & Monotonic Stack & Next greater, histogram problems & Binary Search (14-day), Priority Queues (7-day), Special Algorithms (1-day) & $\square$ \\
\hline
Day 24 & Topological Sort & Kahn's algorithm, dependencies & DFS (14-day), Union-Find (7-day), Monotonic Stack (1-day) & $\square$ \\
\hline
Day 25 & Graph Algorithms & Shortest path, MST & BFS (14-day), Trie (7-day), Topological Sort (1-day) & $\square$ \\
\hline
Day 26 & Advanced Graph & Strongly connected components & DP (14-day), Advanced Trie (7-day), Graph Algorithms (1-day) & $\square$ \\
\hline
Day 27 & Mock Interview 1 & Timed practice, pattern recognition & Week 1 Integration (14-day), Matrix (7-day), Advanced Graph (1-day) & $\square$ \\
\hline
Day 28 & Mock Interview 2 & Final assessment, weak areas & All patterns that showed weakness in Mock 1 & $\square$ \\
\hline
\end{tabular}
\end{table}

\subsubsection{Daily Review Strategy}

\begin{itemize}
\item \textbf{1-Day Review:} Quick template rewriting (5-10 minutes)
\item \textbf{3-Day Review:} Solve 1 easy problem using the pattern (15-20 minutes)
\item \textbf{7-Day Review:} Solve 1 medium problem, focus on edge cases (25-30 minutes)
\item \textbf{14-Day Review:} Solve 1 challenging problem or combo pattern (35-40 minutes)
\end{itemize}

\textbf{Total Daily Time Investment:} 90-120 minutes (30-45 min new learning + 60-75 min reviews)

\newpage

\subsection{Daily Routine Checklist}

\begin{center}
\textbf{\Large Daily Practice Checklist}\\
\textbf{Date: \_\_\_\_\_\_\_\_\_\_\_\_\_\_\_\_\_\_}
\end{center}

\subsubsection{Morning Routine (30 minutes)}
\begin{itemize}
\item[$\square$] \textbf{Template Writing (10 min):} Write 3 random templates from memory
  \begin{itemize}
  \item Template 1: \_\_\_\_\_\_\_\_\_\_\_\_\_\_\_\_\_\_\_\_\_\_\_\_\_\_\_ Time: \_\_\_\_\_ min
  \item Template 2: \_\_\_\_\_\_\_\_\_\_\_\_\_\_\_\_\_\_\_\_\_\_\_\_\_\_\_ Time: \_\_\_\_\_ min
  \item Template 3: \_\_\_\_\_\_\_\_\_\_\_\_\_\_\_\_\_\_\_\_\_\_\_\_\_\_\_ Time: \_\_\_\_\_ min
  \end{itemize}
\item[$\square$] \textbf{Pattern Recognition (10 min):} Look at problem titles and identify patterns
\item[$\square$] \textbf{Complexity Analysis (10 min):} Review time/space complexity for each pattern
\end{itemize}

\subsubsection{Evening Session (60-90 minutes)}
\begin{itemize}
\item[$\square$] \textbf{Problem Solving (45-60 min):} 2-3 problems focusing on current week's patterns
  \begin{itemize}
  \item Problem 1: \_\_\_\_\_\_\_\_\_\_\_\_\_\_\_\_\_\_\_\_\_\_\_\_\_\_\_\_\_\_\_\_\_\_\_\_\_\_\_\_\_\_\_\_\_\_\_\_\_\_\_\_\_\_\_\_\_\_\_\_\_\_\_
  \item Problem 2: \_\_\_\_\_\_\_\_\_\_\_\_\_\_\_\_\_\_\_\_\_\_\_\_\_\_\_\_\_\_\_\_\_\_\_\_\_\_\_\_\_\_\_\_\_\_\_\_\_\_\_\_\_\_\_\_\_\_\_\_\_\_\_
  \item Problem 3: \_\_\_\_\_\_\_\_\_\_\_\_\_\_\_\_\_\_\_\_\_\_\_\_\_\_\_\_\_\_\_\_\_\_\_\_\_\_\_\_\_\_\_\_\_\_\_\_\_\_\_\_\_\_\_\_\_\_\_\_\_\_\_
  \end{itemize}
\item[$\square$] \textbf{Template Review (15-30 min):} Review and rewrite struggled templates
\end{itemize}

\subsubsection{Weekly Reflection}
\textbf{What went well this week?}\\
\fbox{\begin{minipage}{0.9\textwidth}
\vspace{1cm}
\end{minipage}}

\textbf{What needs improvement?}\\
\fbox{\begin{minipage}{0.9\textwidth}
\vspace{1cm}
\end{minipage}}

\textbf{Next week's focus:}\\
\fbox{\begin{minipage}{0.9\textwidth}
\vspace{1cm}
\end{minipage}}

\newpage

\subsection{Mock Interview Tracker}

\begin{table}[h]
\centering
\begin{tabular}{|p{2cm}|p{3cm}|p{4cm}|p{2cm}|p{3cm}|}
\hline
\textbf{Date} & \textbf{Problem} & \textbf{Pattern Used} & \textbf{Time} & \textbf{Notes/Improvements} \\
\hline
\_\_\_\_\_\_\_\_ & \_\_\_\_\_\_\_\_\_\_\_\_\_\_\_\_\_\_\_\_\_\_ & \_\_\_\_\_\_\_\_\_\_\_\_\_\_\_\_\_\_\_\_\_\_ & \_\_\_\_\_ & \_\_\_\_\_\_\_\_\_\_\_\_\_\_\_\_\_\_\_\_\_\_\_\_\_\_\_\_\_\_\_\_\_\_ \\
\hline
\_\_\_\_\_\_\_\_ & \_\_\_\_\_\_\_\_\_\_\_\_\_\_\_\_\_\_\_\_\_\_ & \_\_\_\_\_\_\_\_\_\_\_\_\_\_\_\_\_\_\_\_\_\_ & \_\_\_\_\_ & \_\_\_\_\_\_\_\_\_\_\_\_\_\_\_\_\_\_\_\_\_\_\_\_\_\_\_\_\_\_\_\_\_\_ \\
\hline
\_\_\_\_\_\_\_\_ & \_\_\_\_\_\_\_\_\_\_\_\_\_\_\_\_\_\_\_\_\_\_ & \_\_\_\_\_\_\_\_\_\_\_\_\_\_\_\_\_\_\_\_\_\_ & \_\_\_\_\_ & \_\_\_\_\_\_\_\_\_\_\_\_\_\_\_\_\_\_\_\_\_\_\_\_\_\_\_\_\_\_\_\_\_\_ \\
\hline
\_\_\_\_\_\_\_\_ & \_\_\_\_\_\_\_\_\_\_\_\_\_\_\_\_\_\_\_\_\_\_ & \_\_\_\_\_\_\_\_\_\_\_\_\_\_\_\_\_\_\_\_\_\_ & \_\_\_\_\_ & \_\_\_\_\_\_\_\_\_\_\_\_\_\_\_\_\_\_\_\_\_\_\_\_\_\_\_\_\_\_\_\_\_\_ \\
\hline
\_\_\_\_\_\_\_\_ & \_\_\_\_\_\_\_\_\_\_\_\_\_\_\_\_\_\_\_\_\_\_ & \_\_\_\_\_\_\_\_\_\_\_\_\_\_\_\_\_\_\_\_\_\_ & \_\_\_\_\_ & \_\_\_\_\_\_\_\_\_\_\_\_\_\_\_\_\_\_\_\_\_\_\_\_\_\_\_\_\_\_\_\_\_\_ \\
\hline
\end{tabular}
\end{table}

\subsubsection{Interview Readiness Checklist}

\textbf{Before Your Real Interview:}
\begin{itemize}
\item[$\square$] Can write all 12 main templates from memory in under 2 minutes each
\item[$\square$] Completed at least 100 problems across all difficulty levels
\item[$\square$] Can identify correct pattern within 30 seconds of reading problem
\item[$\square$] Solve medium problems consistently in 15-20 minutes
\item[$\square$] Explain time/space complexity confidently for all solutions
\item[$\square$] Completed at least 5 timed mock interview sessions
\item[$\square$] Comfortable with edge cases and boundary conditions
\item[$\square$] Can clearly communicate approach before coding
\end{itemize}

\textbf{Final Preparation Notes:}\\
\fbox{\begin{minipage}{0.9\textwidth}
\vspace{2cm}
\end{minipage}}

\end{document}