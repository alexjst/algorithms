\documentclass[8pt,landscape]{article}
\usepackage{multicol}
\usepackage{calc}
\usepackage{ifthen}
\usepackage[landscape]{geometry}
\usepackage{amsmath,amsthm,amsfonts,amssymb}
\usepackage{color,graphicx,overpic}
\usepackage{hyperref}
\usepackage{listings}
\usepackage{xcolor}
\usepackage{fancyhdr}

\geometry{top=.3in,left=.3in,right=.3in,bottom=.3in,footskip=10pt}

% Configure page style with page numbers
\pagestyle{fancy}
\fancyhf{} % clear all header and footer fields
\fancyfoot[C]{\scriptsize\thepage} % center page number in footer
\renewcommand{\headrulewidth}{0pt} % remove header line
\renewcommand{\footrulewidth}{0pt} % remove footer line
\fancypagestyle{plain}{%
  \fancyhf{}
  \fancyfoot[C]{\scriptsize\thepage}
  \renewcommand{\headrulewidth}{0pt}
  \renewcommand{\footrulewidth}{0pt}
}

% Define colors for code
\definecolor{codegreen}{rgb}{0,0.6,0}
\definecolor{codegray}{rgb}{0.5,0.5,0.5}
\definecolor{codepurple}{rgb}{0.58,0,0.82}
\definecolor{backcolour}{rgb}{0.95,0.95,0.92}

\lstdefinestyle{mystyle}{
    backgroundcolor=\color{backcolour},
    commentstyle=\color{codegreen},
    keywordstyle=\color{magenta},
    numberstyle=\scriptsize\color{codegray},
    stringstyle=\color{codepurple},
    basicstyle=\ttfamily\scriptsize,
    breakatwhitespace=false,
    breaklines=true,
    captionpos=b,
    keepspaces=true,
    numbersep=3pt,
    showspaces=false,
    showstringspaces=false,
    showtabs=false,
    tabsize=2
}

\lstset{style=mystyle}

\begin{document}
\raggedright
\scriptsize
\begin{multicols}{3}

\setlength{\premulticols}{1pt}
\setlength{\postmulticols}{1pt}
\setlength{\multicolsep}{1pt}
\setlength{\columnsep}{2pt}

\begin{center}
     \Large{\underline{Python Algorithm Templates}} \\
     \small{25 Essential Patterns for Coding Interviews}
\end{center}

\section*{Helper Classes}
\begin{lstlisting}[language=Python]
# Standard data structures (used throughout)
class ListNode:
    def __init__(self, val=0, next=None):
        self.val = val
        self.next = next

class TreeNode:
    def __init__(self, val=0, left=None, right=None):
        self.val = val
        self.left = left
        self.right = right
\end{lstlisting}

\section*{1. Two Pointers}
\textbf{Use when:} Sorted array, pair/triplet sums, palindromes \\
\textbf{Time:} O(n) \quad \textbf{Space:} O(1)
\begin{lstlisting}[language=Python]
# Opposite ends pattern
def two_sum_sorted(arr, target):
    left, right = 0, len(arr) - 1
    while left < right:
        current = arr[left] + arr[right]
        if current == target:
            return [left, right]
        elif current < target:
            left += 1
        else:
            right -= 1
    return []

# Same direction (fast/slow)
def remove_duplicates(arr):
    slow = 0
    for fast in range(1, len(arr)):
        if arr[fast] != arr[slow]:
            slow += 1
            arr[slow] = arr[fast]
    return slow + 1
\end{lstlisting}

\section*{2. Sliding Window}
\textbf{Use when:} Substring/subarray with constraints \\
\textbf{Time:} O(n) \quad \textbf{Space:} O(k)
\begin{lstlisting}[language=Python]
# Fixed size window
def max_sum_subarray(arr, k):
    window_sum = sum(arr[:k])
    max_sum = window_sum
    for i in range(k, len(arr)):
        window_sum += arr[i] - arr[i-k]
        max_sum = max(max_sum, window_sum)
    return max_sum

# Variable size window
def longest_substring_k_distinct(s, k):
    char_count = {}
    left = 0
    max_len = 0
    for right in range(len(s)):
        char_count[s[right]] = char_count.get(s[right], 0) + 1
        while len(char_count) > k:
            char_count[s[left]] -= 1
            if char_count[s[left]] == 0:
                del char_count[s[left]]
            left += 1
        max_len = max(max_len, right - left + 1)
    return max_len
\end{lstlisting}

\section*{3. Binary Search}
\textbf{Use when:} Sorted array, find first/last, optimization \\
\textbf{Time:} O(log n) \quad \textbf{Space:} O(1)
\begin{lstlisting}[language=Python]
# Standard binary search
def binary_search(arr, target):
    left, right = 0, len(arr) - 1
    while left <= right:
        mid = left + (right - left) // 2
        if arr[mid] == target:
            return mid
        elif arr[mid] < target:
            left = mid + 1
        else:
            right = mid - 1
    return -1

# Find first occurrence
def find_first(arr, target):
    left, right = 0, len(arr) - 1
    result = -1
    while left <= right:
        mid = left + (right - left) // 2
        if arr[mid] == target:
            result = mid
            right = mid - 1  # Continue searching left
        elif arr[mid] < target:
            left = mid + 1
        else:
            right = mid - 1
    return result

# Search space binary search
# Tip: Binary search on the ANSWER range, not input array
def min_capacity(weights, days):
    left, right = max(weights), sum(weights)
    while left < right:
        mid = left + (right - left) // 2
        if can_ship(weights, days, mid):
            right = mid
        else:
            left = mid + 1
    return left
\end{lstlisting}

\section*{4. Slow and Fast Pointers}
\textbf{Use when:} Linked list cycles, find middle \\
\textbf{Time:} O(n) \quad \textbf{Space:} O(1)
\begin{lstlisting}[language=Python]
# Cycle detection
def has_cycle(head):
    slow = fast = head
    while fast and fast.next:
        slow = slow.next
        fast = fast.next.next
        if slow == fast:
            return True
    return False

# Find cycle start
def detect_cycle(head):
    slow = fast = head
    while fast and fast.next:
        slow = slow.next
        fast = fast.next.next
        if slow == fast:
            slow = head
            while slow != fast:
                slow = slow.next
                fast = fast.next
            return slow
    return None

# Find middle
def find_middle(head):
    slow = fast = head
    while fast and fast.next:
        slow = slow.next
        fast = fast.next.next
    return slow
\end{lstlisting}

\section*{5. Linked List Reversal}
\textbf{Use when:} Reverse list/partial, reorder \\
\textbf{Time:} O(n) \quad \textbf{Space:} O(1)
\begin{lstlisting}[language=Python]
# Reverse entire list
def reverse_list(head):
    prev, curr = None, head
    while curr:
        next_temp = curr.next
        curr.next = prev
        prev = curr
        curr = next_temp
    return prev

# Reverse between positions m and n
def reverse_between(head, m, n):
    dummy = ListNode(0)
    dummy.next = head
    prev = dummy
    for _ in range(m - 1):
        prev = prev.next

    curr = prev.next
    for _ in range(n - m):
        temp = curr.next
        curr.next = temp.next
        temp.next = prev.next
        prev.next = temp
    return dummy.next
\end{lstlisting}

\section*{6. Binary Tree Traversal}
\textbf{Use when:} Tree navigation, level processing \\
\textbf{Time:} O(n) \quad \textbf{Space:} O(h) recursive, O(n) iterative
\begin{lstlisting}[language=Python]
# Recursive inorder
def inorder(root):
    if not root:
        return []
    return inorder(root.left) + [root.val] + inorder(root.right)

# Iterative inorder
def inorder_iterative(root):
    result, stack = [], []
    curr = root
    while curr or stack:
        while curr:
            stack.append(curr)
            curr = curr.left
        curr = stack.pop()
        result.append(curr.val)
        curr = curr.right
    return result

# Level order (BFS)
from collections import deque
def level_order(root):
    if not root:
        return []
    result, queue = [], deque([root])
    while queue:
        level = []
        for _ in range(len(queue)):
            node = queue.popleft()
            level.append(node.val)
            if node.left:
                queue.append(node.left)
            if node.right:
                queue.append(node.right)
        result.append(level)
    return result
\end{lstlisting}

\section*{7. DFS (Depth-First Search)}
\textbf{Use when:} Tree/graph traversal, connectivity \\
\textbf{Time:} O(V+E) \quad \textbf{Space:} O(V)
\begin{lstlisting}[language=Python]
# Recursive DFS
def dfs(node, visited, graph):
    if node in visited:
        return
    visited.add(node)
    for neighbor in graph[node]:
        dfs(neighbor, visited, graph)

# Iterative DFS
def dfs_iterative(start, graph):
    stack = [start]
    visited = set()
    while stack:
        node = stack.pop()
        if node not in visited:
            visited.add(node)
            for neighbor in graph[node]:
                if neighbor not in visited:
                    stack.append(neighbor)
    return visited

# Count connected components
def count_components(n, edges):
    graph = {i: [] for i in range(n)}
    for u, v in edges:
        graph[u].append(v)
        graph[v].append(u)

    visited = set()
    count = 0
    for i in range(n):
        if i not in visited:
            dfs(i, visited, graph)
            count += 1
    return count
\end{lstlisting}

\section*{8. BFS (Breadth-First Search)}
\textbf{Use when:} Shortest path, level-order \\
\textbf{Time:} O(V+E) \quad \textbf{Space:} O(V)
\begin{lstlisting}[language=Python]
# Standard BFS
from collections import deque
def bfs(start, graph):
    queue = deque([start])
    visited = set([start])
    while queue:
        node = queue.popleft()
        for neighbor in graph[node]:
            if neighbor not in visited:
                visited.add(neighbor)
                queue.append(neighbor)
    return visited

# Multi-source BFS (matrix)
def bfs_matrix(matrix):
    queue = deque()
    for i in range(len(matrix)):
        for j in range(len(matrix[0])):
            if matrix[i][j] == target:
                queue.append((i, j))

    directions = [(0,1), (1,0), (0,-1), (-1,0)]
    while queue:
        x, y = queue.popleft()
        for dx, dy in directions:
            nx, ny = x + dx, y + dy
            if 0 <= nx < len(matrix) and 0 <= ny < len(matrix[0]):
                # Process neighbor
                pass

# Shortest path with distance
def shortest_path(start, end, graph):
    queue = deque([(start, 0)])
    visited = {start}
    while queue:
        node, dist = queue.popleft()
        if node == end:
            return dist
        for neighbor in graph[node]:
            if neighbor not in visited:
                visited.add(neighbor)
                queue.append((neighbor, dist + 1))
    return -1
\end{lstlisting}

\section*{9. Dynamic Programming}
\textbf{Use when:} Optimization, overlapping subproblems \\
\textbf{Time:} O(n) to O(n\textsuperscript{3}) \quad \textbf{Space:} O(n) to O(n\textsuperscript{2})
\begin{lstlisting}[language=Python]
# Top-down (Memoization)
# IMPORTANT: Avoid mutable default arguments!
def fib_memo(n, memo=None):
    if memo is None:
        memo = {}
    if n in memo:
        return memo[n]
    if n <= 1:
        return n
    memo[n] = fib_memo(n-1, memo) + fib_memo(n-2, memo)
    return memo[n]

# Bottom-up (Tabulation)
def fib_tab(n):
    if n <= 1:
        return n
    dp = [0] * (n + 1)
    dp[1] = 1
    for i in range(2, n + 1):
        dp[i] = dp[i-1] + dp[i-2]
    return dp[n]

# House robber pattern
def rob(nums):
    if not nums:
        return 0
    prev2 = 0
    prev1 = nums[0]
    for i in range(1, len(nums)):
        temp = max(prev1, prev2 + nums[i])
        prev2 = prev1
        prev1 = temp
    return prev1

# Coin change (unbounded knapsack)
def coin_change(coins, amount):
    dp = [float('inf')] * (amount + 1)
    dp[0] = 0
    for i in range(1, amount + 1):
        for coin in coins:
            if i >= coin:
                dp[i] = min(dp[i], dp[i-coin] + 1)
    return dp[amount] if dp[amount] != float('inf') else -1

# Kadane's Algorithm (Maximum Subarray Sum)
# Classic DP pattern - often asked by name in interviews
def max_subarray_sum(nums):
    if not nums:
        return 0
    max_so_far = nums[0]
    current_max = nums[0]
    for num in nums[1:]:
        current_max = max(num, current_max + num)
        max_so_far = max(max_so_far, current_max)
    return max_so_far
\end{lstlisting}

\section*{10. Backtracking}
\textbf{Use when:} Generate all solutions, CSP \\
\textbf{Time:} Exponential \quad \textbf{Space:} O(n)
\begin{lstlisting}[language=Python]
# Permutations
def permute(nums):
    result = []
    def backtrack(path, remaining):
        if not remaining:
            result.append(path[:])
            return
        for i in range(len(remaining)):
            path.append(remaining[i])
            backtrack(path, remaining[:i] + remaining[i+1:])
            path.pop()
    backtrack([], nums)
    return result

# Subsets
def subsets(nums):
    result = []
    def backtrack(start, path):
        result.append(path[:])
        for i in range(start, len(nums)):
            path.append(nums[i])
            backtrack(i + 1, path)
            path.pop()
    backtrack(0, [])
    return result

# Combinations (choose k elements)
def combine(nums, k):
    result = []
    def backtrack(start, path):
        if len(path) == k:
            result.append(path[:])
            return
        for i in range(start, len(nums)):
            path.append(nums[i])
            backtrack(i + 1, path)
            path.pop()
    backtrack(0, [])
    return result

# Word search in grid
def exist(board, word):
    def backtrack(i, j, k):
        if k == len(word):
            return True
        if (i < 0 or i >= len(board) or j < 0 or
            j >= len(board[0]) or board[i][j] != word[k]):
            return False
        temp = board[i][j]
        board[i][j] = '#'  # Mark visited
        result = (backtrack(i+1, j, k+1) or
                 backtrack(i-1, j, k+1) or
                 backtrack(i, j+1, k+1) or
                 backtrack(i, j-1, k+1))
        board[i][j] = temp  # Restore
        return result

    for i in range(len(board)):
        for j in range(len(board[0])):
            if backtrack(i, j, 0):
                return True
    return False
\end{lstlisting}

\section*{11. Bit Manipulation}
\textbf{Use when:} Set operations, unique elements \\
\textbf{Time:} O(1) per operation \quad \textbf{Space:} O(1)
\begin{lstlisting}[language=Python]
# Common operations
x & (x - 1)         # Clear rightmost set bit
x & -x              # Get rightmost set bit
x | (1 << i)        # Set bit i
x & ~(1 << i)       # Clear bit i
x ^ (1 << i)        # Toggle bit i
(x >> i) & 1        # Check if bit i is set
x & (x - 1) == 0    # Check if power of 2

# Find single number (XOR)
def single_number(nums):
    result = 0
    for num in nums:
        result ^= num
    return result

# Count set bits
def count_bits(n):
    count = 0
    while n:
        n &= n - 1
        count += 1
    return count

# Subset generation using bits
def subsets_bits(nums):
    n = len(nums)
    result = []
    for mask in range(1 << n):
        subset = []
        for i in range(n):
            if (mask >> i) & 1:
                subset.append(nums[i])
        result.append(subset)
    return result
\end{lstlisting}

\section*{12. Prefix Sum}
\textbf{Use when:} Range queries, subarray sums \\
\textbf{Time:} O(n) build, O(1) query \quad \textbf{Space:} O(n)
\begin{lstlisting}[language=Python]
# 1D prefix sum
class PrefixSum:
    def __init__(self, nums):
        self.prefix = [0] * (len(nums) + 1)
        for i in range(len(nums)):
            self.prefix[i+1] = self.prefix[i] + nums[i]

    def range_sum(self, left, right):
        return self.prefix[right+1] - self.prefix[left]

# Subarray sum equals k
def subarray_sum(nums, k):
    count = 0
    prefix_sum = 0
    sum_freq = {0: 1}
    for num in nums:
        prefix_sum += num
        if prefix_sum - k in sum_freq:
            count += sum_freq[prefix_sum - k]
        sum_freq[prefix_sum] = sum_freq.get(prefix_sum, 0) + 1
    return count

# 2D prefix sum
class Matrix2D:
    def __init__(self, matrix):
        m, n = len(matrix), len(matrix[0])
        self.prefix = [[0] * (n+1) for _ in range(m+1)]
        for i in range(1, m+1):
            for j in range(1, n+1):
                self.prefix[i][j] = (matrix[i-1][j-1] +
                                    self.prefix[i-1][j] +
                                    self.prefix[i][j-1] -
                                    self.prefix[i-1][j-1])

    def region_sum(self, r1, c1, r2, c2):
        return (self.prefix[r2+1][c2+1] -
                self.prefix[r1][c2+1] -
                self.prefix[r2+1][c1] +
                self.prefix[r1][c1])
\end{lstlisting}

\section*{13. Top K Elements / Heaps}
\textbf{Use when:} Priority problems, k largest/smallest \\
\textbf{Time:} O(n log k) \quad \textbf{Space:} O(k)
\begin{lstlisting}[language=Python]
import heapq

# K largest elements (use min-heap)
def k_largest(nums, k):
    heap = []
    for num in nums:
        heapq.heappush(heap, num)
        if len(heap) > k:
            heapq.heappop(heap)
    return heap

# K smallest elements (use max-heap with negation)
def k_smallest(nums, k):
    heap = []
    for num in nums:
        heapq.heappush(heap, -num)
        if len(heap) > k:
            heapq.heappop(heap)
    return [-x for x in heap]

# Merge k sorted lists
def merge_k_lists(lists):
    heap = []
    for i, lst in enumerate(lists):
        if lst:
            heapq.heappush(heap, (lst.val, i, lst))

    dummy = ListNode(0)
    curr = dummy
    while heap:
        val, i, node = heapq.heappop(heap)
        curr.next = node
        curr = curr.next
        if node.next:
            heapq.heappush(heap, (node.next.val, i, node.next))
    return dummy.next

# Running median
class MedianFinder:
    def __init__(self):
        self.small = []  # max-heap (negated)
        self.large = []  # min-heap

    def addNum(self, num):
        heapq.heappush(self.small, -num)
        heapq.heappush(self.large, -heapq.heappop(self.small))
        if len(self.large) > len(self.small):
            heapq.heappush(self.small, -heapq.heappop(self.large))

    def findMedian(self):
        if len(self.small) > len(self.large):
            return -self.small[0]
        return (-self.small[0] + self.large[0]) / 2
\end{lstlisting}

\section*{14. Monotonic Stack}
\textbf{Use when:} Next greater/smaller element \\
\textbf{Time:} O(n) \quad \textbf{Space:} O(n)
\begin{lstlisting}[language=Python]
# Next greater element
def next_greater(nums):
    stack = []  # Monotonic decreasing
    result = [-1] * len(nums)
    for i, num in enumerate(nums):
        while stack and nums[stack[-1]] < num:
            idx = stack.pop()
            result[idx] = num
        stack.append(i)
    return result

# Next smaller element
def next_smaller(nums):
    stack = []  # Monotonic increasing
    result = [-1] * len(nums)
    for i, num in enumerate(nums):
        while stack and nums[stack[-1]] > num:
            idx = stack.pop()
            result[idx] = num
        stack.append(i)
    return result

# Largest rectangle in histogram
def largest_rectangle(heights):
    stack = []
    max_area = 0
    heights.append(0)
    for i, h in enumerate(heights):
        while stack and heights[stack[-1]] > h:
            height = heights[stack.pop()]
            width = i if not stack else i - stack[-1] - 1
            max_area = max(max_area, height * width)
        stack.append(i)
    return max_area
\end{lstlisting}

\section*{15. Overlapping Intervals}
\textbf{Use when:} Scheduling, merging intervals \\
\textbf{Time:} O(n log n) \quad \textbf{Space:} O(n)
\begin{lstlisting}[language=Python]
# Merge intervals
def merge(intervals):
    intervals.sort(key=lambda x: x[0])
    merged = [intervals[0]]
    for current in intervals[1:]:
        if current[0] <= merged[-1][1]:
            merged[-1][1] = max(merged[-1][1], current[1])
        else:
            merged.append(current)
    return merged

# Insert interval
def insert(intervals, new_interval):
    result = []
    i = 0
    # Before overlap
    while i < len(intervals) and intervals[i][1] < new_interval[0]:
        result.append(intervals[i])
        i += 1
    # Merge overlapping
    while i < len(intervals) and intervals[i][0] <= new_interval[1]:
        new_interval[0] = min(new_interval[0], intervals[i][0])
        new_interval[1] = max(new_interval[1], intervals[i][1])
        i += 1
    result.append(new_interval)
    # After overlap
    result.extend(intervals[i:])
    return result

# Minimum intervals to remove
def erase_overlap_intervals(intervals):
    intervals.sort(key=lambda x: x[1])
    count = 0
    end = float('-inf')
    for interval in intervals:
        if interval[0] >= end:
            end = interval[1]
        else:
            count += 1
    return count
\end{lstlisting}

\section*{16. Trie (Prefix Tree)}
\textbf{Use when:} Prefix matching, autocomplete \\
\textbf{Time:} O(m) per operation \quad \textbf{Space:} O(total chars)
\begin{lstlisting}[language=Python]
class TrieNode:
    def __init__(self):
        self.children = {}
        self.is_end = False

class Trie:
    def __init__(self):
        self.root = TrieNode()

    def insert(self, word):
        node = self.root
        for char in word:
            if char not in node.children:
                node.children[char] = TrieNode()
            node = node.children[char]
        node.is_end = True

    def search(self, word):
        node = self.root
        for char in word:
            if char not in node.children:
                return False
            node = node.children[char]
        return node.is_end

    def starts_with(self, prefix):
        node = self.root
        for char in prefix:
            if char not in node.children:
                return False
            node = node.children[char]
        return True

# Word search II - Advanced application combining Trie + DFS
# Note: Core Trie template above, this shows practical usage
def find_words(board, words):
    trie = Trie()
    for word in words:
        trie.insert(word)

    result = set()
    def dfs(i, j, node, path):
        if node.is_end:
            result.add(path)
        if (i < 0 or i >= len(board) or j < 0 or
            j >= len(board[0]) or board[i][j] not in node.children):
            return
        char = board[i][j]
        board[i][j] = '#'
        for di, dj in [(0,1), (1,0), (0,-1), (-1,0)]:
            dfs(i+di, j+dj, node.children[char], path+char)
        board[i][j] = char

    for i in range(len(board)):
        for j in range(len(board[0])):
            dfs(i, j, trie.root, "")
    return list(result)
\end{lstlisting}

\section*{17. Union-Find (Disjoint Set)}
\textbf{Use when:} Connectivity, components \\
\textbf{Time:} O($\alpha$(n)) $\approx$ O(1) \quad \textbf{Space:} O(n)
\begin{lstlisting}[language=Python]
class UnionFind:
    def __init__(self, n):
        self.parent = list(range(n))
        self.rank = [0] * n
        self.components = n

    def find(self, x):
        # Path compression optimization
        if self.parent[x] != x:
            self.parent[x] = self.find(self.parent[x])
        return self.parent[x]

    def union(self, x, y):
        # Union by rank optimization
        root_x, root_y = self.find(x), self.find(y)
        if root_x != root_y:
            if self.rank[root_x] < self.rank[root_y]:
                self.parent[root_x] = root_y
            elif self.rank[root_x] > self.rank[root_y]:
                self.parent[root_y] = root_x
            else:
                self.parent[root_y] = root_x
                self.rank[root_x] += 1
            self.components -= 1
            return True
        return False

    def connected(self, x, y):
        return self.find(x) == self.find(y)

# Count components
def count_components(n, edges):
    uf = UnionFind(n)
    for u, v in edges:
        uf.union(u, v)
    return uf.components

# Kruskal's MST
def minimum_spanning_tree(n, edges):
    uf = UnionFind(n)
    edges.sort(key=lambda x: x[2])  # Sort by weight
    mst = []
    for u, v, weight in edges:
        if uf.union(u, v):
            mst.append((u, v, weight))
            if len(mst) == n - 1:
                break
    return mst
\end{lstlisting}

\section*{18. Greedy Algorithms}
\textbf{Use when:} Local optimum leads to global \\
\textbf{Time:} Varies \quad \textbf{Space:} O(1) typically
\begin{lstlisting}[language=Python]
# Activity selection
def activity_selection(activities):
    activities.sort(key=lambda x: x[1])  # Sort by end
    result = [activities[0]]
    last_end = activities[0][1]
    for start, end in activities[1:]:
        if start >= last_end:
            result.append((start, end))
            last_end = end
    return result

# Jump game
def can_jump(nums):
    max_reach = 0
    for i in range(len(nums)):
        if i > max_reach:
            return False
        max_reach = max(max_reach, i + nums[i])
        if max_reach >= len(nums) - 1:
            return True
    return True

# Gas station
def can_complete_circuit(gas, cost):
    if sum(gas) < sum(cost):
        return -1
    tank = 0
    start = 0
    for i in range(len(gas)):
        tank += gas[i] - cost[i]
        if tank < 0:
            tank = 0
            start = i + 1
    return start
\end{lstlisting}

\section*{19. Advanced DP}
\textbf{2D DP, State Machines, DP on Trees}
\begin{lstlisting}[language=Python]
# 2D DP: Edit distance
def min_distance(word1, word2):
    m, n = len(word1), len(word2)
    dp = [[0] * (n+1) for _ in range(m+1)]
    for i in range(m+1):
        dp[i][0] = i
    for j in range(n+1):
        dp[0][j] = j
    for i in range(1, m+1):
        for j in range(1, n+1):
            if word1[i-1] == word2[j-1]:
                dp[i][j] = dp[i-1][j-1]
            else:
                dp[i][j] = 1 + min(dp[i-1][j],    # Delete
                                   dp[i][j-1],    # Insert
                                   dp[i-1][j-1])  # Replace
    return dp[m][n]

# State machine DP: Stock with cooldown
def max_profit_cooldown(prices):
    held = -prices[0]  # Holding stock
    sold = 0           # Just sold (cooldown)
    rest = 0           # Resting (can buy)
    for price in prices[1:]:
        new_held = max(held, rest - price)
        new_sold = held + price
        new_rest = max(rest, sold)
        held, sold, rest = new_held, new_sold, new_rest
    return max(sold, rest)

# DP on trees: House robber III
def rob_tree(root):
    def dfs(node):
        if not node:
            return (0, 0)  # (rob, not_rob)
        left_rob, left_not = dfs(node.left)
        right_rob, right_not = dfs(node.right)
        rob = node.val + left_not + right_not
        not_rob = max(left_rob, left_not) + max(right_rob, right_not)
        return (rob, not_rob)
    return max(dfs(root))
\end{lstlisting}

\section*{20. Graph Algorithms}
\textbf{Dijkstra, Floyd-Warshall, MST}
\begin{lstlisting}[language=Python]
# Dijkstra's shortest path
import heapq
def dijkstra(graph, start):
    distances = {node: float('inf') for node in graph}
    distances[start] = 0
    pq = [(0, start)]
    while pq:
        curr_dist, curr = heapq.heappop(pq)
        if curr_dist > distances[curr]:
            continue
        for neighbor, weight in graph[curr]:
            distance = curr_dist + weight
            if distance < distances[neighbor]:
                distances[neighbor] = distance
                heapq.heappush(pq, (distance, neighbor))
    return distances

# Floyd-Warshall (all-pairs shortest)
# Assumes edges is a list of [u, v, weight]
def floyd_warshall(n, edges):
    dist = [[float('inf')] * n for _ in range(n)]
    for i in range(n):
        dist[i][i] = 0
    for u, v, weight in edges:
        dist[u][v] = weight

    for k in range(n):
        for i in range(n):
            for j in range(n):
                dist[i][j] = min(dist[i][j],
                                dist[i][k] + dist[k][j])
    return dist

# Bellman-Ford (handles negative weights)
# Returns distances dict or None if negative cycle exists
def bellman_ford(n, edges, start):
    dist = [float('inf')] * n
    dist[start] = 0

    # Relax all edges n-1 times
    for _ in range(n - 1):
        for u, v, weight in edges:
            if dist[u] != float('inf') and dist[u] + weight < dist[v]:
                dist[v] = dist[u] + weight

    # Check for negative cycles
    for u, v, weight in edges:
        if dist[u] != float('inf') and dist[u] + weight < dist[v]:
            return None  # Negative cycle detected

    return dist

# Prim's MST
def prim_mst(n, edges):
    graph = {i: [] for i in range(n)}
    for u, v, w in edges:
        graph[u].append((v, w))
        graph[v].append((u, w))

    mst = []
    visited = set([0])
    edges_heap = graph[0][:]
    heapq.heapify(edges_heap)

    while edges_heap and len(visited) < n:
        weight, u, v = heapq.heappop(edges_heap)
        if v not in visited:
            visited.add(v)
            mst.append((u, v, weight))
            for neighbor, w in graph[v]:
                if neighbor not in visited:
                    heapq.heappush(edges_heap, (w, v, neighbor))
    return mst
\end{lstlisting}

\section*{21. Topological Sort}
\textbf{Use when:} DAG ordering, dependencies \\
\textbf{Time:} O(V+E) \quad \textbf{Space:} O(V)
\begin{lstlisting}[language=Python]
# Kahn's algorithm (BFS-based)
from collections import deque, defaultdict
def topological_sort_kahn(n, edges):
    graph = defaultdict(list)
    in_degree = [0] * n
    for u, v in edges:
        graph[u].append(v)
        in_degree[v] += 1

    queue = deque([i for i in range(n) if in_degree[i] == 0])
    result = []
    while queue:
        node = queue.popleft()
        result.append(node)
        for neighbor in graph[node]:
            in_degree[neighbor] -= 1
            if in_degree[neighbor] == 0:
                queue.append(neighbor)

    return result if len(result) == n else []

# DFS-based topological sort
def topological_sort_dfs(n, edges):
    graph = defaultdict(list)
    for u, v in edges:
        graph[u].append(v)

    visited = set()
    rec_stack = set()
    result = []

    def dfs(node):
        if node in rec_stack:
            return False  # Cycle
        if node in visited:
            return True
        visited.add(node)
        rec_stack.add(node)
        for neighbor in graph[node]:
            if not dfs(neighbor):
                return False
        rec_stack.remove(node)
        result.append(node)
        return True

    for i in range(n):
        if i not in visited:
            if not dfs(i):
                return []
    return result[::-1]

# Course schedule with prerequisites
def can_finish(num_courses, prerequisites):
    graph = defaultdict(list)
    in_degree = [0] * num_courses
    for course, prereq in prerequisites:
        graph[prereq].append(course)
        in_degree[course] += 1

    queue = deque([i for i in range(num_courses)
                   if in_degree[i] == 0])
    count = 0
    while queue:
        course = queue.popleft()
        count += 1
        for next_course in graph[course]:
            in_degree[next_course] -= 1
            if in_degree[next_course] == 0:
                queue.append(next_course)

    return count == num_courses
\end{lstlisting}

\section*{22. Cyclic Sort}
\textbf{Use when:} Array with numbers in range [1, n] \\
\textbf{Time:} O(n) \quad \textbf{Space:} O(1)
\begin{lstlisting}[language=Python]
# Core cyclic sort pattern
def cyclic_sort(nums):
    i = 0
    while i < len(nums):
        j = nums[i] - 1  # Target index for nums[i]
        if 0 <= j < len(nums) and nums[i] != nums[j]:
            nums[i], nums[j] = nums[j], nums[i]
        else:
            i += 1
    return nums

# Find missing number (array contains [0, n-1])
def find_missing_number(nums):
    # Cyclic sort to place each number at its index
    i = 0
    while i < len(nums):
        j = nums[i]
        if j < len(nums) and nums[i] != nums[j]:
            nums[i], nums[j] = nums[j], nums[i]
        else:
            i += 1
    # Find first index that doesn't match
    for i in range(len(nums)):
        if nums[i] != i:
            return i
    return len(nums)

# Find all duplicates (array contains [1, n])
def find_duplicates(nums):
    duplicates = []
    i = 0
    while i < len(nums):
        j = nums[i] - 1
        if nums[i] != nums[j]:
            nums[i], nums[j] = nums[j], nums[i]
        else:
            i += 1
    # Numbers not at correct index are duplicates
    for i in range(len(nums)):
        if nums[i] != i + 1:
            duplicates.append(nums[i])
    return duplicates
\end{lstlisting}

\section*{23. Lowest Common Ancestor (LCA)}
\textbf{Use when:} Finding shared ancestor in a tree \\
\textbf{Time:} O(n) \quad \textbf{Space:} O(h)
\begin{lstlisting}[language=Python]
# LCA in binary tree (classic recursive pattern)
def lowest_common_ancestor(root, p, q):
    if not root or root == p or root == q:
        return root

    left = lowest_common_ancestor(root.left, p, q)
    right = lowest_common_ancestor(root.right, p, q)

    if left and right:
        return root
    return left or right

# LCA in BST (optimized using BST property)
def lca_bst(root, p, q):
    while root:
        if p.val < root.val and q.val < root.val:
            root = root.left
        elif p.val > root.val and q.val > root.val:
            root = root.right
        else:
            return root
\end{lstlisting}

\section*{24. Matrix Traversal}
\textbf{Use when:} Spiral traversal, in-place rotation \\
\textbf{Time:} O(m*n) \quad \textbf{Space:} O(1)
\begin{lstlisting}[language=Python]
# Spiral order traversal
def spiral_order(matrix):
    if not matrix:
        return []
    res = []
    left, right = 0, len(matrix[0]) - 1
    top, bottom = 0, len(matrix) - 1

    while left <= right and top <= bottom:
        for i in range(left, right + 1):
            res.append(matrix[top][i])
        top += 1
        for i in range(top, bottom + 1):
            res.append(matrix[i][right])
        right -= 1
        if top <= bottom:
            for i in range(right, left - 1, -1):
                res.append(matrix[bottom][i])
            bottom -= 1
        if left <= right:
            for i in range(bottom, top - 1, -1):
                res.append(matrix[i][left])
            left += 1
    return res

# Rotate matrix 90 degrees clockwise (in-place)
def rotate(matrix):
    n = len(matrix)
    # Transpose
    for i in range(n):
        for j in range(i, n):
            matrix[i][j], matrix[j][i] = matrix[j][i], matrix[i][j]
    # Reverse each row
    for i in range(n):
        matrix[i].reverse()
\end{lstlisting}

\section*{Sorting Algorithm Reference}
\begin{tabular}{|l|l|l|l|l|}
\hline
\textbf{Algorithm} & \textbf{Time (Avg)} & \textbf{Time (Worst)} & \textbf{Space} & \textbf{Notes} \\
\hline
Merge Sort & O(n log n) & O(n log n) & O(n) & Stable, good for linked lists \\
Quick Sort & O(n log n) & O(n\textsuperscript{2}) & O(log n) & In-place, unstable \\
Timsort (Python) & O(n log n) & O(n log n) & O(n) & Stable, hybrid merge+insertion \\
\hline
\end{tabular}

\section*{Pattern Recognition Guide}
\begin{itemize}
\item \textbf{Sorted array + pairs:} Two Pointers
\item \textbf{Substring with constraints:} Sliding Window
\item \textbf{Search in sorted:} Binary Search
\item \textbf{Linked list cycles:} Fast/Slow Pointers
\item \textbf{Tree traversal:} DFS/BFS/Preorder/Inorder
\item \textbf{Tree ancestor problems:} LCA
\item \textbf{Optimization + overlapping:} Dynamic Programming
\item \textbf{Max subarray sum:} Kadane's Algorithm
\item \textbf{All solutions:} Backtracking
\item \textbf{Choose k elements:} Combinations
\item \textbf{Range queries:} Prefix Sum
\item \textbf{Priority-based:} Heap
\item \textbf{Next greater/smaller:} Monotonic Stack
\item \textbf{Intervals:} Sort + Merge/Greedy
\item \textbf{Prefix matching:} Trie
\item \textbf{Connectivity:} Union-Find
\item \textbf{DAG ordering:} Topological Sort
\item \textbf{Array with numbers 1 to n:} Cyclic Sort
\item \textbf{Spiral/rotation matrix:} Matrix Traversal
\item \textbf{Negative edge weights:} Bellman-Ford
\end{itemize}

\section*{Complexity Quick Reference}
\begin{tabular}{|l|l|l|}
\hline
\textbf{Pattern} & \textbf{Time} & \textbf{Space} \\
\hline
Two Pointers & O(n) & O(1) \\
Sliding Window & O(n) & O(k) \\
Binary Search & O(log n) & O(1) \\
Fast/Slow Pointers & O(n) & O(1) \\
DFS/BFS & O(V+E) & O(V) \\
DP (1D) & O(n) & O(n) \\
DP (2D) & O(n\textsuperscript{2}) & O(n\textsuperscript{2}) \\
Kadane's & O(n) & O(1) \\
Backtracking & Exponential & O(n) \\
Prefix Sum & O(n) build, O(1) query & O(n) \\
Heap & O(n log k) & O(k) \\
Monotonic Stack & O(n) & O(n) \\
Intervals & O(n log n) & O(n) \\
Trie & O(m) & O(total) \\
Union-Find & O($\alpha$(n)) & O(n) \\
Topological Sort & O(V+E) & O(V) \\
Cyclic Sort & O(n) & O(1) \\
LCA & O(n) & O(h) \\
Matrix Traversal & O(m*n) & O(1) \\
Bellman-Ford & O(V*E) & O(V) \\
\hline
\end{tabular}

\end{multicols}
\end{document}
